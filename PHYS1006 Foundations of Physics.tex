\documentclass{extarticle}
\usepackage[utf8]{inputenc}

\usepackage[8pt]{extsizes}
\usepackage{geometry}
\geometry{a4paper, portrait, margin=0.5cm, bottom=1.5cm}
%, right=5cm

\usepackage{fancyhdr}
\pagestyle{fancy}
\fancyfoot[L]{\jobname\hspace{1cm}  Page \thepage}
\fancyfoot[C]{}
\fancyfoot[R]{}

%\documentclass[20pt]{extarticle}
%\documentclass[a5paper,11pt]{article}
%\documentclass[6pt]{extarticle}
%\documentclass[14pt]{extarticle}
%\documentclass[a3paper,11pt]{article}


%\usepackage[a3paper,margin=0.1in]{geometry}
%\usepackage[a3paper]{geometry}

\usepackage{verbatim}
\usepackage{mathptmx}
\usepackage{amsfonts}
\usepackage{amssymb}
\usepackage{gensymb}
%\usepackage{extsizes}

\usepackage{multicol}
\usepackage{tabu}




\usepackage[dvipsnames]{xcolor}

\usepackage[most]{tcolorbox}
\tcbuselibrary{breakable}

%\usepackage[subtle]{savetrees}
%\usepackage[moderate]{savetrees}
\usepackage[extreme]{savetrees}

\raggedright
\raggedbottom

\begin{document} 
\begin{multicols}{4}
%\chapter{First Chapter}


%\smallskip
%\medskip
%\bigskip


%gets rid of section counters 
\setcounter{secnumdepth}{0}


\section{Describing Motion:
Kinematics in One Dimension}

\subsection{Displacement}
“the change in x,” or “change in position” 

$\Delta x = x_2 - x_1$

The change in any quantity means the final value of that quantity, minus the initial value.

\subsection{Average Speed}
The average speed of an object is defined as the total distance travelled along its path divided by the time it takes to travel this distance:

$\text{average speed} = \frac{\text{total distance travelled}}{\text{time elapsed}}$

\subsection{Average Velocity}
The average velocity is defined in terms of displacement, rather than total distance travelled:

$\text{average velocity} = \frac{ \text{displacement} }{ \text{time elapsed} } = \frac{ \text{final position} - \text{initial position} }{ \text{time elapsed} }$



Then the average velocity, defined as the displacement divided by the elapsed time

$\bar{v} = \frac{x_2 - x_1}{t_2 - t_1} = \frac{\Delta x}{\Delta t}$


\subsection{Instantaneous Velocity}
instantaneous velocity at any moment is defined as the average velocity over an infinitesimally short time interval

$v = \lim_{\Delta t \to 0} \frac{\Delta x}{\Delta t}$

\subsection{Average Acceleration}
Average acceleration is defined as the change in velocity divided by the time taken to make this change:

$\text{average acceleration} = \frac{\text{change of velocity}}{\text{time elapsed}}$

$\bar{a} = \frac{v_2 - v_1}{t_2 - t_1} = \frac{\Delta v}{\Delta t}$

\subsection{Instantaneous Acceleration}
The instantaneous acceleration, a, can be defined in analogy to instantaneous
velocity as the average acceleration over an infinitesimally short time interval at
a given instant:

$a = \lim_{\Delta t \to 0} \frac{\Delta v}{\Delta t}$


\subsection{Deceleration}
We have a deceleration whenever the magnitude of the velocity is decreasing;

thus the velocity and acceleration point in opposite directions when there is deceleration.


















\section{C}

\subsection{Operator precedence}

\begin{tabular}{|l|l|l|l|}
\hline
Precedence     & Operator                                              & Description                                                            & Associativity \\ \hline
\end{tabular}


\begin{tabular}{|l|l|l|l|}
\hline
     &                                               &                                                             &  \\ \hline
1              & ++ --                                                 & Suffix/postfix increment and decrement                                 & LR            \\
               & ()                                                    & Function call                                                          &               \\
               & {[}{]}                                                & Array subscripting                                                     &               \\
               & .                                                     & Structure and union member access                                      &               \\
               & -\textgreater{}                                       & Structure and union member access through pointer                      &               \\
               & (type)\{list\}                                        & Compound literal(C99)                                                  &               \\ \hline
2              & ++ --                                                 & Prefix increment and decrement{[}note 1{]}                             & RL            \\
               & + -                                                   & Unary plus and minus                                                   &               \\
               & ! $\sim$                                              & Logical NOT and bitwise NOT                                            &               \\
               & (type)                                                & Type cast                                                              &               \\
               & *                                                     & Indirection (dereference)                                              &               \\
               & \&                                                    & Address-of                                                             &               \\
               & sizeof                                                & Size-of{[}note 2{]}                                                    &               \\
               & \_Alignof                                             & Alignment requirement(C11)                                             &               \\ \hline
3              & * / \%                                                & Multiplication, division, and remainder                                & LR            \\ \hline
4              & + -                                                   & Addition and subtraction                                               &               \\ \hline
5              & \textless{}\textless \textgreater{}\textgreater{}     & Bitwise left shift and right shift                                     &               \\ \hline
6              & \textless \textless{}=                                & For relational operators \textless and \textless{}= respectively       &               \\ \hline
               & \textgreater \textgreater{}=                          & For relational operators \textgreater and \textgreater{}= respectively &               \\ \hline
7              & == !=                                                 & For relational = and != respectively                                   &               \\ \hline
8              & \&                                                    & Bitwise AND                                                            &               \\ \hline
9              & \textasciicircum{}                                    & Bitwise XOR (exclusive or)                                             &               \\ \hline
10             & |                                                     & Bitwise OR (inclusive or)                                              &               \\ \hline
11             & \&\&                                                  & Logical AND                                                            &               \\ \hline
12             & ||                                                    & Logical OR                                                             &               \\ \hline
13             & ?:                                                    & Ternary conditional{[}note 3{]}                                        & RL            \\ \hline
14 & =                                                     & Simple assignment                                                      &               \\
               & += -=                                                 & Assignment by sum and difference                                       &               \\
               & *= /= \%=                                             & Assignment by product, quotient, and remainder                         &               \\
               & \textless{}\textless{}= \textgreater{}\textgreater{}= & Assignment by bitwise left shift and right shift                       &               \\
               & \&= \textasciicircum{}= |=                            & Assignment by bitwise AND, XOR, and OR                                 &               \\ \hline
15             & ,                                                     & Comma                                                                  & LR            \\ \hline
\end{tabular}


\subsubsection{1s}

Debugging Tips for Program Systems

A list of suggestions for debugging a program system follows.

1. Carefully document each function parameter and local variable using comments as you write the code. Also, describe the function’s purpose using comments.

2. Create a trace of execution by displaying the function name as you enter it.

3. Trace or display the values of all input and input/output parameters upon entry to a function. Check that these values make sense.

4. Trace or display the values of all function outputs after returning from a function. Verify that these values are correct by hand computation. Make sure you declare all input/output and output parameters as pointer types.

5. Make sure that a function stub assigns a value to the variable pointed to by each output parameter.



If you are using a debugger, you may be able to specify whether you want to execute a function as if it were a single statement or whether you want to step through the individual statements of a function. Initially, execute the function as a single statement and trace the values of all input and output parameters (tips 3 and 4 above). If the function results are incorrect, step through its individual statements.

If you are not using a debugger, you should plan for debugging as you write each function rather than waiting until after you finish the whole program. Include the display statements (mentioned in debugging tips 2 through 4) in the original C code for the function. When you are satisfied that the function works correctly, remove the debugging statements. The simplest way is to change them to comments by enclosing them within the symbols /* , */ ; if you have a problem later, you can remove these symbols, thereby changing the comments back to executable statements.








\begin{verbatim}



/*
* Reduces a fraction by dividing its numerator and denominator by their
* greatest common divisor.
*/
void
reduce_fraction(int *nump, /* input/output - */
                int *denomp) /* numerator and denominator of fraction */
{
    int gcd; /* greatest common divisor of numerator & denominator */

    gcd = find_gcd(*nump, *denomp);
    *nump = *nump / gcd;
    *denomp = *denomp / gcd;
}


\end{verbatim}










\section{Latex tools}

https://www.tablesgenerator.com/latex\_tables

http://w2.syronex.com/jmr/latex-symbols-converter

\subsection{Adding and subtracting}

Let $z_1=3+2i$ and $z_2=4+i$ then

$z_1+z_2=3+2i+4+i$
$=\left(3+4\right)+\left(2+1\right)i$
$=7+3i$


We add/subtract the real and imaginary parts.
In general

$\left(a+ib\right)\pm\left(c+id\right)=\left(a\pm c\right)+i\left(b\pm d\right)$

\subsection{Multiplication}

$\left(3+2i\right)\times\left(4+i\right)$ $=3\times4+3\times i+2i\times4+2i^2$ $=12+3i+8i-2$
since $i^2=-1$

$=10+11i$

Note that the product of two complex numbers is another complex number.


\subsection{Complex conjugates}

The complex conjugate of a complex number $z=a+ib$ is $\overline{z}=a-ib$. 

(That is, change the sign of the imaginary part).

Example: Let $z=1+2i$ then $\overline{z}=1-2i$. Now consider

$z\overline{z}=\left(1+2i\right)\left(1-2i\right)$

$=1-2i+2i-4i^2\Rightarrow\ \ =1-4\left(-1\right)$

$=1+4=5$, a pure real number.



\subsection{Division of complex numbers}

$z=\frac{3+i}{1-i}\Rightarrow\frac{3+i}{1-i}\times\frac{1+i}{1+i}\Rightarrow\frac{\left(3+i\right)\left(1+i\right)}{\left(1-i\right)\left(1+i\right)}=\frac{3+3i+i+i^2}{1+i-i-i^2}=\frac{3+3i+i+(-1)}{1+i-i-(-1)}=\frac{2+4i}{2}=1+2i$

In general, $z=\frac{z_1}{z_2}$, multiply by $\frac{{\overline{z}}_2}{{\overline{z}}_2}$.

\subsection{Polar form of complex numbers}

$r$ is called the magnitude of the complex number $z, r=\left|z\right|=\sqrt{x^2+y^2}$

$\theta$ is called the argument of the complex number $z$, $\theta=arg\left(z\right)=arctan{\left(y/x\right)}$

\begin{tabu}{|X|X|}
    \hline
$\frac{\pi}{2}<\theta<\pi$ & $0<\theta<\frac{\pi}{2}$ \\ \hline
$\pi<\theta<\frac{3\pi}{2}$ & $\frac{3\pi}{2}<\theta<2\pi$ \\ \hline
    \end{tabu}



%$\pi/2<\theta<\pi$ \qquad $	0<\theta<\pi/2$

%$\pi<\theta<3\pi/2$ \qquad $3\pi/2<\theta<2\pi$



\subsection{Convert to Cartesian form}

(look for rad and find the cos and sin)(if unsure just use a calculator)

$2cis\left(\frac{3\pi}{4}\right)=2cos{\left(\frac{3\pi}{4}\right)}+2isin{\left(\frac{3\pi}{4}\right)}$

$=-2\frac{1}{\sqrt2}+2i\frac{1}{\sqrt2}=-\sqrt2+i\sqrt2$

\subsection{Multiplication of complex numbers in Polar form}

$z_1=r_1cis\left(\theta_1\right)$ and $z_2=r_2cis\left(\theta_2\right)$

$z_1\ast z_2=r_1\ast r_2cis\left(\theta_1+\theta_2\right)$

\subsection{Division of complex numbers in Polar form}

$\frac{z_1}{z_2}=\frac{r_1}{r_2}cis\left(\theta_1-\theta_2\right)$

\subsection{Powers of complex numbers in Polar form}

$z=rcis\left(\theta\right)$

$z^n=r^ncis\left(n\theta\right)$


\subsection{Fractional Powers using De Moivre's theorem}

$\left(cos\theta+isin\theta\right)^n=\left(e^{i\theta}\right)^n=e^{in\theta}=cos\left(n\theta\right)+isin\left(n\theta\right)$
for any real number $n$.


\subsection{Finding all roots of numbers.}

First write the complex number in polar form

$z=rcis\left(\theta\right)$


$z^\frac{1}{q}=r^\frac{1}{q}cis\left(\frac{\theta}{q}+k\frac{2\pi}{q}\right)$ $k=0,1,2,\ldots,q-1$

Convert to Cartesian form

1. All roots have the same magnitude.

2. The nth roots of a complex number z are separated by$\ \frac{2\pi}{n}$ and there are n of them.

Complex roots always appear in conjugate pairs.



\section{Trigonometry Ratio Table}
    \begin{tabu}{|X|X|X|X|X|}
    \hline
D				& R					& sin				& cos						& tan						\\ \hline
$0\degree$	& $0$					& $0$				& $1$						& $0$						\\ \hline
$30\degree$	& $\frac{\pi}{6}$	& $\frac{1}{2}$	& $\frac{\sqrt{3}}{2}$	& $\frac{1}{\sqrt{3}}$	\\ \hline
$45\degree$	& $\frac{\pi}{4}$	& $\frac{1}{\sqrt{2}}$	& $\frac{1}{\sqrt{2}}$	& $1$	\\ \hline
$60\degree$	& $\frac{\pi}{3}$	& $\frac{\sqrt{3}}{2}$	& $\frac{1}{2}$	& $\sqrt{3}$	\\ \hline
$90\degree$	& $\frac{\pi}{2}$	& $1$	& $0$	& $ND$	\\ \hline
$120\degree$	& $\frac{2\pi}{3}$	& $\frac{\sqrt{3}}{2}$	& $-\frac{1}{2}$	& $ $	\\ \hline
$135\degree$	& $\frac{3\pi}{4}$	& $\frac{\sqrt{2}}{2}$	& $-\frac{\sqrt{2}}{2}$	& $ $	\\ \hline
$150\degree$	& $\frac{5\pi}{6}$	& $\frac{1}{2}$	& $-\frac{\sqrt{3}}{2}$	& $ $	\\ \hline
D				& R					& sin				& cos						& tan						\\ \hline
$180\degree$	& $\pi$					& $0$				& $-1$						& $0$						\\ \hline
$210\degree$	& $\frac{7\pi}{6}$	& $-\frac{1}{2}$	& $-\frac{\sqrt{3}}{2}$	& $ $	\\ \hline
$225\degree$	& $\frac{5\pi}{4}$	& $-\frac{\sqrt{2}}{2}$	& $-\frac{\sqrt{2}}{2}$	& $ $	\\ \hline
$240\degree$	& $\frac{4\pi}{3}$	& $-\frac{\sqrt{3}}{2}$	& $-\frac{1}{2}$	& $ $	\\ \hline
$270\degree$	& $\frac{3\pi}{2}$	& $-1$	& $0$	& $ND$	\\ \hline
$300\degree$	& $\frac{5\pi}{3}$	& $-\frac{\sqrt{3}}{2}$	& $\frac{1}{2}$	& $ $	\\ \hline
$315\degree$	& $\frac{7\pi}{4}$	& $-\frac{\sqrt{2}}{2}$	& $\frac{\sqrt{2}}{2}$	& $ $	\\ \hline
$330\degree$	& $\frac{11\pi}{6}$	& $-\frac{1}{2}$	& $\frac{\sqrt{3}}{2}$	& $ $	\\ \hline


$360\degree$ & $2\pi$ & $0$ & $1$ & $ $ \\ \hline
    \end{tabu}



\section{new stuff.}

\subsection{Work done by a force}


Note to andrew: in this case force is the magnitude of the force in the direction of displacement (like the horizontal component of the force) Which you get using scalar projection


Work = force $\times$ displacement

= magnitude of $F$ in direction of $s \times ||s||$

= scalar projection of $F$ on $s \times ||s||$

$= ||F||\cos{(\theta)}||s||$

$= F.s$

Work $= F.s$


$=\vec{F} \cdot \hat{\vec{s}} \times ||\vec{s}||$

$=\frac{\vec{F} \cdot \vec{s}}{||\vec{s}||} \times ||\vec{s}||$

So

$=\vec{F} \cdot \vec{s}$

$= ||\vec{F}||\cos(\theta)||\vec{s}||$


\subsection{cross product}
$\vec{a}\times \vec{b}$ gives $\vec{c}$

which is perpendicular to both $\vec{a}$ and $\vec{b}$

with a direction given by the right-hand rule

and a magnitude equal to the area of the parallelogram that the vectors span.

\subsection{area of the parallelogram}
$area =||\vec{u}||\cos{(\theta)}||\vec{v}||$

$area =||\vec{a}\times \vec{b}||$
\subsection{triangle}
half of a parallelogram is a triangle

$area =\frac{1}{2}||\vec{a}\times \vec{b}||$

\subsection{Scalar triple product}
Three vectors defining a parallelepiped

$volume=\vec{a}\cdot(\vec{b}\times \vec{c})$
if $volume=0$ then the vectors are coplanar

\section{random stuff.}
\section{Vectors}

\begin{tcolorbox}[enhanced jigsaw,sharp corners,coltext=black,colback=Red!25!white,boxrule=0pt,breakable,size=minimal]
The unit vector a hat gives the direction cosines of a

$[cos a,cos b,cos c]=[\frac{a_1}{|a|},\frac{a_2}{|a|},\frac{a_3}{|a|}]= a hat$

and from that we get the direction angles of a

$a=cos^{-1}(\frac{a_1}{|a|}), b=cos^{-1}(\frac{a_2}{|a|}), c=cos^{-1}(\frac{a_3}{|a|})$
\end{tcolorbox}

\subsection{Angle between the vectors}

If theta is the angle between the vectors a and b, then

$a\cdot b=\left|a\right|\left|b\right|\cos{\theta}$

If theta is the angle between the non-zero vectors a and b, then

$\cos{\theta}=\frac{a\cdot b}{\left|a\right|\left|b\right|}$

Note: This works in 3-space because there are still only 2 vectors, they just have 3 components.

\subsection{Orthogonal / Perpendicular vectors}

$a\cdot b=\left|a\right|\left|b\right|\cos{\left(\pi/2\right)}=0$

If a dot b = 0 then theta = 90 degrees

Trick: in 2-space given x,y then both -y,x and y,-x are perpendicular.

\subsection{Scalar projection}

$\vec{u}\cdot\hat{v}=\frac{\vec{u}\cdot\vec{v}}{\left|\vec{v}\right|}$

Note: the vector we're projecting onto has the unit vector


\subsection{Vector projection}

Multiplying the distance (scalar projection of u and v) with the direction of v (unit vector of v)

$\left(\vec{u}\cdot\hat{v}\right)\hat{v}=\frac{\vec{u}\cdot\vec{v}}{\left|\vec{v}\right|^2}\vec{v}$

\subsection{Shortest distance between a point and a line}

$P$ is a point on the line

$\vec{u}$ is the lines direction vector

$Q$ is the random point

$distance=\sqrt{(Q-P)^2-  (\frac{(Q-P)\cdot\vec{u}}{\left|\vec{u}\right|})^2}$


\subsection{line intersections}
Note if $\vec{a} = m\vec{b}$ then the two lines are parallel.


%https://www.uplifteducation.org/cms/lib/TX01001293/Centricity/Domain/273/COMMON%20QUESTIONS%20-%20VECTORS.pdf
Worked example

Line 1: $x=1-t$ , $y=t$ , $z=3-2t$

Line 2: $x=1+2u$, $y=-1-u$, $z=4+3u$

Show that lines 1 and 2 intersect and find angle between them

for x $1-t =1+2u \Rightarrow t =-2u$,

for y $t =-1-u \Rightarrow$ $ -2u=-1-u \Rightarrow u=1 \& t =-2$

checking with z: $3-2t =4+3u \Rightarrow 3-2(-2)=4+3(1)$ confirmed 



\subsection{Vector Subspaces}
A subset $U$ of ${\mathbb{R}}^n$ is a subspace of ${\mathbb{R}}^n$ i if and only if

(a) for any $u, v \in U, u + v \in U$, and

(b) for any $u \in U$ and any scalar $s, su \in U$.

We say U is a subspace of ${\mathbb{R}}^n $ if it is closed under addition and scalar multiplication.

%\smallskip
%\medskip
%\bigskip
\smallskip
\smallskip



For any ${\mathbb{R}}^n $, ${\mathbb{R}}^n $ itself and $\{0\}$ are subspaces.

$\{0\}$ is called the zero vector space, simply denoted as 0.



Ex: Note that for any ${\mathbb{R}}^n $, ${\mathbb{R}}^n $ itself and $\{0\}$ are subspaces. $\{0\}$ is called the zero vector space, simply denoted as 0.

\begin{tcolorbox}[enhanced jigsaw,sharp corners,coltext=black,colback=BurntOrange!25!white,boxrule=0pt,breakable,size=minimal]



Ex: Let $U$ denote all vectors in ${\mathbb{R}}^3 $ such that their second component is 1. Show that $U$ is not a subspace of ${\mathbb{R}}^3 $.

Soln: Note that $0 = [0, 0, 0]$ does not have its second component equal to 1, i.e. it is not in $U$. Thus, since $U$ does not contain the zero vector, it cannot be a subspace of ${\mathbb{R}}^3 $.

Alternatively, $a = [0, 1, 2]$ and $b = [1, 1, 1]$ are both in $U$, but $a+b = [1, 2, 3] \notin U$, so the set is not closed under addition and it is thus not a subspace.


\end{tcolorbox}


Ex: Show that the set of vectors in ${\mathbb{R}}^3 $ where the third component equals the sum of the first two components is a subspace of ${\mathbb{R}}^3 $.

Soln: The set can be written as

$U=\left\{
{\left[\begin{matrix}x_1\\x_2\\x_3\\\end{matrix}\right]}
\in \mathbb{R}^3
\vert
x_3=x_1 + x_2
 \right\}$

Let $u = [u_1, u_2, u_3]$ and $v = [v_1, v_2, v_3]$ be in U and let s be any scalar. Then $u, v \in U$ means $u_3 = u_1 + u_2$ and $v_3 = v_1 + v_2$.

Closure under addition. 

Now consider $u+v = [u_1+v_1, u_2+v_2, u_3+v_3]$.

Notice that $u_3+v_3 = (u_1+u_2)+(v_1+v_2) = (u_1 + v_1) + (u_2 + v_2)$ which shows that $u + v \in U$, i.e. $U$ is closed under addition.

Closure under multiplication.

Next, consider $su = [su_1, su_2, su_3]$.

Note that $su_3 = s(u_1 + u_2) = su_1 + su_2$, i.e. $su \in U$ as well and U is therefore closed under scalar multiplication.

Combining these two properties, U must be a subspace of ${\mathbb{R}}^3$.



Note that $0 \in U$, as required.





\subsection{Solution Space of Homogeneous System}
The solution to a homogeneous system set of linear equations is a subspace
\begin{tcolorbox}[enhanced jigsaw,sharp corners,coltext=black,colback=BurntOrange!25!white,boxrule=0pt,breakable,size=minimal]
Consider a homogeneous system of m linear equations in n unknowns

$Ax=0$

i.e. $A$ is $m \times n$ and $x \in{\mathbb{R}}^n$. Let

$V=\left\{
x
\in \mathbb{R}^n
\vert
Ax=0
 \right\}$


i.e. the set of all possible solutions of the system.

Let $u, v \in V$ , then $Au = 0$ and $Av = 0$. Furthermore, let $s$ be any scalar.

Then $A(u + v) = Au + Av = 0 + 0 = 0$ which show that $u + v \in V$, i.e. $V$ is closed under addition.

Similarly, $A(su) = sAu = s0 = 0$ which shows that $su \in V$ , i.e. $V$ is closed under scalar multiplication.
Hence, $V$ is a vector subspace of ${\mathbb{R}}^n$ and therefore a vector space. In this sense, we refer to $V$ as the null space of the matrix A.
\end{tcolorbox}


\section{Planes}
\subsection{The vector equation of the plane}
we have: $n\cdot\left(r-r_0\right)=0$

That can also be written as: $n\cdot r=n\cdot r_0$

$a\left(x-x_0\right)+b\left(y-y_0\right)+c\left(z-z_0\right)=0$

$ax+by+cz=d$

Note: a, b and c are still the normal and you can read the normal out


\begin{tcolorbox}[enhanced jigsaw,sharp corners,coltext=black,colback=Red!25!white,boxrule=0pt,breakable,size=minimal]
\subsection{Shortest distance between a point and a plane}
$A$ is a point on the plane

$\vec{n}$ is the planes normal vector

$P$ is the random point

$distance=\frac{\vec{AP}\cdot\vec{n}}{\left|\vec{n}\right|}$



\end{tcolorbox}







\subsection{Matrix Inverses}
The Matrix Inverse of a 2X2

if $A=\left[\begin{matrix}a&b\\c&d\\\end{matrix}\right]\rightarrow A^{-1}=\frac{1}{detA}\left[\begin{matrix}d&-b\\-c&a\\\end{matrix}\right]$

where $det\ A=ad-bc$

and $det\ A\neq0$

\subsection{Linear Combinations}
Let $\left\{u_1,u_2,... u_m\right\}\subset{\mathbb{R}}^n$. If the vector $u$ can be expressed in the form

$u = c_1u_1+c_2u_2+...+ c_mu_m$

for some scalars $c_1,c_2,... c_m\in\mathbb{R}$, we say that $u$ is a linear combination of $u_1,u_2,... u_m$.

Clearly, $u$ is itself a vector in ${\mathbb{R}}^n$.

\begin{tcolorbox}[enhanced jigsaw,sharp corners,coltext=black,colback=BurntOrange!25!white,boxrule=0pt,breakable,size=minimal]

Ex: 

Let 
$u=\left[\begin{matrix}1\\2\\-1\\\end{matrix}\right]$,$v=\left[\begin{matrix}6\\4\\2\\\end{matrix}\right]$ $\in{\mathbb{R}}^3$.
Show that
$w_1=\left[\begin{matrix}9\\2\\7\\\end{matrix}\right]$
is a linear combination of $u$ and $v$, while $w_2=\left[\begin{matrix}4\\-1\\8\\\end{matrix}\right]$ is not.



Soln: In general, we need to solve a system of linear equations to address this type of problem. For the first vector,


$c_1u+c_2v=w_1$ gives 
$c_1\left[\begin{matrix}1\\2\\-1\\\end{matrix}\right]+c_2\left[\begin{matrix}6\\4\\2\\\end{matrix}\right]=\left[\begin{matrix}9\\2\\7\\\end{matrix}\right]$

Make an augmented Matrix and solve with row reduction

If it's consistent then $w_1$ can be written as a linear combination

If the system is inconsistent and has no solution then $w_2$ can not be written as a linear combination of $u$ and $v$.

\end{tcolorbox}

(from weekly workshop)
Worked example

Let 
$v_1=\left[\begin{matrix}1\\1\\0\\\end{matrix}\right]$,$v_2=\left[\begin{matrix}0\\1\\1\\\end{matrix}\right]$ and $v_3=\left[\begin{matrix}1\\1\\1\\\end{matrix}\right]$$\in{\mathbb{R}}^3$.
Show that
$w=\left[\begin{matrix}1\\0\\0\\\end{matrix}\right]$
is a linear combination of $v_1$, $v_2$ and $v_3$.

Need to show that

$c_1v_1+c_2v_2+ c_3v_3=w$

$c_1\left[\begin{matrix}1\\1\\0\\\end{matrix}\right]+c_2\left[\begin{matrix}0\\1\\1\\\end{matrix}\right]+ c_3\left[\begin{matrix}1\\1\\1\\\end{matrix}\right]=\left[\begin{matrix}1\\0\\0\\\end{matrix}\right]$

Solve system of linear equations, $c_1$, $c_2$ and $c_3$.

$1c_1+0c_2+1c_3=1$

$1c_1+1c_2+1c_3=0$

$0c_1+1c_2+1c_3=0$

use gaussian elimination


$\begin{matrix}\left[\begin{matrix}1&0&1\\1&1&1\\0&1&1\\\end{matrix}\middle|\begin{matrix}1\\0\\0\\\end{matrix}\right]&\vec{\begin{matrix}\\R_2=R_2-R_1\\ \\\end{matrix}}\\\end{matrix}$

$\begin{matrix}\left[\begin{matrix}1&0&1\\0&1&0\\0&1&1\\\end{matrix}\middle|\begin{matrix}1\\-1\\0\\\end{matrix}\right]&\vec{\begin{matrix}\\ \\R_3=R_3-R_2\\\end{matrix}}\\\end{matrix}$

$\begin{matrix}\left[\begin{matrix}1&0&1\\0&1&0\\0&0&1\\\end{matrix}\middle|\begin{matrix}1\\-1\\1\\\end{matrix}\right]\\\end{matrix}$STOP

$r(A)=3=r(A|b)=n=3$

$\therefore$ Unique solution

$c_3=1,c_2=-1,c_1=0$

$\therefore 0v_1-1v_2+1v_3=w \therefore$ l.c.

	%\includegraphics[scale=0.2]{MATH1015_Lecture9_Slides_-121.png}






\subsection{Linear Dependence and Independence}
Set up your vectors as a homogeneous system of linear equations $[A|0]$

Solve for the scalar values $c_1, c_2...... c_m$

If you get a unique trivial solution they are linearly independent

If you get an infinite amount of solutions they are linearly dependent

\begin{tcolorbox}[enhanced jigsaw,sharp corners,coltext=black,colback=Green!25!white,boxrule=0pt,breakable,size=minimal]
So in summary, when testing if a set of vectors is l.i. or l.d. we go through the following steps:

(i) If there only two vectors check to see if they're parallel, i.e. $v_1 = sv_2$.

If they are parallel then they're l.d., else they're l.i.

If there are more than two vectors go to (ii).

(ii) Check to see if the number of vectors m is more than space n (i.e. ${\mathbb{R}}^n)$, 

if $m > n$ then they're l.d.,

if not go to (iii).

(iii) Are the number of vectors m the same as space n, i.e. $m = n$?

If it is then set up matrix A (where the columns of A are the vectors) and calculate the determinant.

If $det (A) = 0$ then they're l.d.,

if $det (A) \ne 0$ then they're l.i.

If number of vectors isn't same as space, i.e. $m \ne n$, go to (iv).

(iv) Set up augmented matrix $[A|0]$ then use E.R.O's to determine the rank of $r(A) =
r(A|0)$.

If $r(A) < m$ then they're l.d.,

if $r(A) = m$ then they're l.i.
\end{tcolorbox}








\section{Basis and Dimension}
\subsection{Quick Notes}
\subsection{Testing if u vector is in V span}

$A = \left|v_1|v_2|...|v_n\right|$

Matrix $[A|u]$ and then solve with row reduction


\subsection{Testing if vectors are a basis for V}

Form the matrix

$A = \left|v_1|v_2|...|v_n\right|$

Then test

They have to be linearly independent and 

every vector in V can be written as a l.c. of $v_i$'s.





\subsection{Find a Basis for span of vectors}

Matrix $A = \left|v_1|v_2|...|v_n\right|$

Reduced row echelon form

Return only non zero rows

If these vectors are l.i., they are a basis for the row space of A





\subsection{Basis of row vectors}

Matrix

Reduced row echelon form

Return only non zero rows

Since these vectors are l.i., they are a basis for the row space of A

Dimension is number of rows




\subsection{Column vectors}

Same

But transpose the matrix

And the return vectors



\subsection{Nullity}

Matrix $[A|0]$
and then solve with row reduction

Reduced row echelon form








\begin{tcolorbox}[enhanced jigsaw,sharp corners,coltext=black,colback=BurntOrange!25!white,boxrule=0pt,breakable,size=minimal]

\subsection{Spanning Sets}
Let $U = \left\{u_1,u_2,...,u_m\right\}$ be a set of vectors in a vector space $V$ . Then span ($U$) is the set of all possible linear combinations of vectors in U, i.e.

$span (U) = \left\{x \in V | x = c_1u_1 + c_2u_2 + ... + c_mu_m \right\}$

where $c_1, c_2,...,c_m$ are scalars.

Ex: Show that $w = \left[ \begin{matrix}3\\5\\1\\\end{matrix} \right]$ is in span ($U$), where $U = \left\{ \left[ \begin{matrix}2\\5\\0\\\end{matrix}\right], \left[ \begin{matrix}2\\0\\2\\\end{matrix}\right] \right\} $.

Soln: We basically need to show that $w$ can be written as a l.c. of the vectors in $U$, i.e. we need to find coecients $c_1$ and $c_2$ such that

$c_1 \left[ \begin{matrix}2\\5\\0\\\end{matrix} \right] + c_2 \left[ \begin{matrix}2\\0\\2\\\end{matrix} \right] = \left[ \begin{matrix}3\\5\\1\\\end{matrix} \right]$
i.e.
$\begin{matrix} 2c_1&+&2c_2&=&3\\&&5c_1&=&5\\&&2c_2&=&1\\ \end{matrix}$

Clearly, these equations can be satised by $c_1 = 1$ and $c_2 = \frac{1}{2}$ . Thus, $w \in$ span ($U$).

\end{tcolorbox}

\begin{tcolorbox}[enhanced jigsaw,sharp corners,coltext=black,colback=BurntOrange!25!white,boxrule=0pt,breakable,size=minimal]

\subsection{Bases}
A set of vectors $\left\{v_1,v_2,...,v_n\right\}$ in a vector space V is called a basis for V if

(i) $\left\{v_1,v_2,...,v_n\right\}$ is linearly independent and 

(ii) $V=span(\left\{v_1,v_2,...,v_n\right\})$, i.e. every vector in V can be written as a l.c. of $v_i$'s.

Note: In view of our previous examples, the problem of determining whether a set of n vectors, $S = \left\{v_1,v_2,...,v_n\right\}$, in ${\mathbb{R}}^n$ forms a basis for ${\mathbb{R}}^n$ or not should be dealt with as follows. Form the square matrix

$A = \left|v_1|v_2|...|v_n\right|$

Then

$det(A) \neq 0$ S does form a basis of ${\mathbb{R}}^n$.

$det(A) = 0$ S does not form a basis of ${\mathbb{R}}^n$.

\end{tcolorbox}



\subsection{Nullspace of a Matrix}
Consider a homogeneous system of m linear equations in n unknowns

Ax = 0

i.e. A is $m \times n$ and $x \in {\mathbb{R}}^n$. Recall that the null space of A, i.e.

$V = \{ x \in {\mathbb{R}}^n | Ax = 0$

is a subspace of ${\mathbb{R}}^n$. The dimension of the null space is called the nullity of A.

\begin{tcolorbox}[enhanced jigsaw,sharp corners,coltext=black,colback=BurntOrange!25!white,boxrule=0pt,breakable,size=minimal]

Hence, $null space(A) = span \left( \left[ \begin{matrix} -1\\-\frac{5}{2}\\1\\2\\\end{matrix} \right], \left[ \begin{matrix} -3\\0\\0\\1\\ \end{matrix} \right] \right)$. Since the two vectors are l.i. (not a scalar multiple of one another), $\left\{ \left[ \begin{matrix} -1\\-\frac{5}{2}\\1\\2\\\end{matrix} \right], \left[ \begin{matrix} -3\\0\\0\\1\\ \end{matrix} \right] \right\}$ is a basis for the null space of A and nullity(A) = 2.


\end{tcolorbox}

\begin{tcolorbox}[enhanced jigsaw,sharp corners,coltext=black,colback=BurntOrange!25!white,boxrule=0pt,breakable,size=minimal]

\subsection{Row Space \& Column Space}
In addition to the nullspace of a matrix A, there are two other important subspaces associated with the matrix A and these are related to the row vectors and column vectors of the matrix A.



dimension = the number of vectors in the basis for the row and column space.

both dimensions and rank are always the same

dim(R)=dim(C)=rank(A).\end{tcolorbox}

























\section{Eigenvectors and eigenvalues}
Some facts:

	The product of the eigenvalues=$det|A|$
	
	The sum of the eigenvalues=trace(A)
	
If you factor the characteristic equation wrong these will be different.


\subsection{Eigenvalues and eigenvectors}

An eigenvector of an $n\times n$ matrix $A$ is a non-zero $n\times 1$ column matrix (vector) $\vec{v}$ with the property:
$A\vec{v}=\lambda\vec{v}$

For some scalar $\lambda$, which we call its eigenvalue.

Examples

Decide if lambda and $\vec{v}$ are eigenvalues and eigenvectors of A respectively

$A=\left[\begin{matrix}1&4\\1&1\\\end{matrix}\right]$	$\lambda=3$	$\vec{v}=\left[\begin{matrix}2\\1\\\end{matrix}\right]$

Decide if lambda and $\vec{v}$ are eigenvalues and eigenvectors of A respectively

$A=\left[\begin{matrix}1&4\\1&1\\\end{matrix}\right]$	$\lambda=-1$	$\vec{v}=\left[\begin{matrix}-2\\1\\\end{matrix}\right]$



\subsection{Finding eigenvalues}
We can rewrite the equation

$A\vec{v}=\lambda\vec{v}$

as

$\left(A-\lambda I\right)\vec{v}=\vec{0}$

where $I$ is the $n\times n$ identity matrix.

This will only have a solution if

$det\left(A-\lambda I\right)=0$

Example: Find the eigenvalues of $A=\left[\begin{matrix}1&4\\1&1\\\end{matrix}\right]$

$det\left(A-\lambda I\right)=0$

$(1-\lambda)(1-\lambda)-(4)(1)$

$1-\lambda-\lambda+\lambda^2-4=\lambda^2-2\lambda-3$ <-- this is the characteristic polynomial

$(\lambda+1)(\lambda-3)=0$

Note: check this by multiplying it out

So the eigenvalues are $\lambda=-1$ and $\lambda=3$


\subsection{Finding eigenvectors}
Once we have found the eigenvalue(s) of A, we substitute these back into

$\left(A-\lambda I\right)\vec{v}=\vec{0}$

and solve for the corresponding eigenvector $\vec{v}$, using row operations.
Note that this will correspond to solving a system of equations with infinitely many solutions.


If $\vec{v}$ is an eigenvector of $A$, then so is every scalar multiple of $\vec{v}$.

When giving a list of possible eigenvectors, we will just give one example for each unique direction: we will not include scalar multiples or combinations of other vectors. We will call this a list of linearly independent eigenvectors.


Example: Find the eigenvectors of $A=\left[\begin{matrix}1&4\\1&1\\\end{matrix}\right]$

From our earlier work, we have that $A-\lambda I=\left[\begin{matrix}1-\lambda&4\\1&1-\lambda\\\end{matrix}\right]$

and we found that the eigenvalues of $A$ are $\lambda=3$ and $\lambda=-1$. We take each eigenvalue in turn and solve the system $\left(A-\lambda I\right)\vec{v}=\vec{0}$

First we consider $\lambda=3$:

$\left[\begin{matrix}1-\lambda&4\\1&1-\lambda\\\end{matrix}\right]\Rightarrow\left[\begin{matrix}1-3&4\\1&1-3\\\end{matrix}\right]\Rightarrow\left[\begin{matrix}-2&4\\1&-2\\\end{matrix}\right]$

Gaussian elimination:

Only reduce row 2

$\left[\begin{matrix}-2&4\\2&-4\\\end{matrix}\right]\Rightarrow\left[\begin{matrix}-2&4\\0&0\\\end{matrix}\right]$

row echelon form:

$\left[\begin{matrix}-2&4\\0&0\\\end{matrix}\right]\Rightarrow\left[\begin{matrix}1&-2\\0&0\\\end{matrix}\right]$

Skipping $\lambda=-1$:

For $\lambda=3$

$1\vec{v_1}-2\vec{v_2}=0$

$1\vec{v_1}=2\vec{v_2}$

If $v_2=t$ then $v_1=2t$

Eigenvector = $\left[\begin{matrix}2t\\t\\\end{matrix}\right] or \left[\begin{matrix}2\\1\\\end{matrix}\right]t$



$\left[\begin{matrix}1&4\\1&1\\\end{matrix}\right]$, eigenvectors: $\left\{\begin{matrix}-2\\1\\\end{matrix}\right\}\leftrightarrow-1$, $\left\{\begin{matrix}2\\1\\\end{matrix}\right\}\leftrightarrow\ 3$

For the 2X2 matrix A: the sum of the eigenvalues equals the trace of the matrix and the product of the eigenvalues equals the determinant we derive then that:

$\lambda^2-\left(trace\left(A\right)\lambda\right)+detA=\mathrm{characteristic\ polynomial}$






\subsection{Cramer’s Rule}


$x_i=\frac{det(A_i)}{det (A)}$ 

where $Ax=b$

where $A_i$ replaces $i$th column in $A$ with $b$


\begin{tcolorbox}[enhanced jigsaw,sharp corners,coltext=black,colback=Red!25!white,boxrule=0pt,breakable,size=minimal]
\subsection{add Eigenspaces?}
\end{tcolorbox}


\begin{tcolorbox}[enhanced jigsaw,sharp corners,coltext=black,colback=Red!25!white,boxrule=0pt,breakable,size=minimal]
\subsection{algebraic multiplicity and geometric multiplicity}

When looking at the eigenstructure of a matrix, the following terminology is useful.

By the algebraic multiplicity (a.m.) of an eigenvalue, we mean the number of times it is repeated as a root of the characteristic equation.

By the geometric multiplicity (g.m.) of an eigenvalue, we mean the dimension of the eigenspace corresponding to it (or, equivalently, the number of l.i. eigenvectors corresponding to it).

e.g. in our second last example,

$\lambda1 = 2$ has $a.m.=2$ and $g.m.=1$, $\lambda2 = 1$ has $a.m.=1$ and $g.m.=1$.

In our last example,

$\lambda1 = 1$ has $a.m.=1$ and $g.m.=1$, $\lambda2 = 5$ has $a.m.=2$ and $g.m.=2$.
\end{tcolorbox}




\begin{tcolorbox}[enhanced jigsaw,sharp corners,coltext=black,colback=BurntOrange!25!white,boxrule=0pt,breakable,size=minimal]
\subsection{template}

\end{tcolorbox}





\end{multicols}






\end{document}
