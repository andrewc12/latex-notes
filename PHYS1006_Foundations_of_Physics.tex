\documentclass{extarticle}
\usepackage[utf8]{inputenc}

%https://tex.stackexchange.com/a/5896
\usepackage{etoolbox}
\newtoggle{editing}
\toggletrue{editing}
%\togglefalse{editing}

%example
\iftoggle{editing}{% editing
}{% finaloutput
}

\iftoggle{editing}{% editing
  %\usepackage[20pt]{extsizes}
  \usepackage{extsizes}
  \usepackage{geometry}
  \geometry{paperwidth=105mm,paperheight=575cm, portrait, margin=0.2cm, bottom=1.5cm} %, right=5cm
  %\geometry{a6paper, portrait, margin=0.2cm, bottom=1.5cm} %, right=5cm
  %\geometry{a4paper, portrait, margin=0.5cm, bottom=1.5cm} %, right=5cm
}{% finaloutput
  \usepackage[8pt]{extsizes}
  \usepackage{geometry}
  \geometry{a4paper, portrait, margin=0.5cm, bottom=1.5cm} %, right=5cm
}



\usepackage{fancyhdr}
\pagestyle{fancy}
\fancyfoot[L]{\jobname\hspace{1cm}  Page \thepage}
\fancyfoot[C]{}
\fancyfoot[R]{}

\usepackage{verbatim}
\usepackage{mathptmx}
\usepackage{amsfonts}
\usepackage{amssymb}
\usepackage{gensymb}

\usepackage{multicol}
%\usepackage{tabu}
\usepackage{tabularx}

\usepackage[dvipsnames]{xcolor}

\usepackage[most]{tcolorbox}
\tcbuselibrary{breakable}

%\usepackage[subtle]{savetrees}
%\usepackage[moderate]{savetrees}
%\usepackage[extreme]{savetrees}

\raggedright
\raggedbottom

%https://tex.stackexchange.com/a/161265
\everymath\expandafter{\the\everymath\displaystyle}





\begin{document} 


\newcommand{\markStart}[1][25]{
\begin{tcolorbox}[enhanced jigsaw,sharp corners,coltext=black,colback=BurntOrange!#1!white,boxrule=pt,breakable,size=minimal]
}

%\markStart[25]
%\markStart

\newcommand{\markEnd}[0]{
\end{tcolorbox}
}




\newcolumntype{Y}{>{\centering\arraybackslash}X}









\iftoggle{editing}{% editing
}{% finaloutput
  \begin{multicols}{3}
}
%\chapter{First Chapter}

%\smallskip
%\medskip
%\bigskip

%gets rid of section counters 
\setcounter{secnumdepth}{0}



\section{Describing Motion:
Kinematics in One Dimension}

\subsection{Displacement}
“the change in x,” or “change in position” 

$\Delta x = x_2 - x_1$

The change in any quantity means the final value of that quantity, minus the initial value.

\subsection{Average Speed}
is defined as the total distance travelled along its path divided by the time it takes to travel this distance.

$\text{average speed} = \frac{\text{total distance travelled}}{\text{time elapsed}}$

\subsection{Average Velocity}
is defined in terms of displacement, rather than total distance travelled.

$\text{average velocity} = \frac{ \text{displacement} }{ \text{time elapsed} } = \frac{ \text{final position} - \text{initial position} }{ \text{time elapsed} }$


is defined as the displacement divided by the elapsed time

$\bar{v} = \frac{x_2 - x_1}{t_2 - t_1} = \frac{\Delta x}{\Delta t}$


\subsection{Instantaneous Velocity}
Instantaneous velocity at any moment is defined as the average velocity over an infinitesimally short time interval.

$v = \lim_{\Delta t \to 0} \frac{\Delta x}{\Delta t}$

\subsection{Average Acceleration}
is defined as the change in velocity divided by the time taken to make this change.

$\text{average acceleration} = \frac{\text{change of velocity}}{\text{time elapsed}}$

$\bar{a} = \frac{v_2 - v_1}{t_2 - t_1} = \frac{\Delta v}{\Delta t}$

\subsection{Instantaneous Acceleration}
Instantaneous acceleration, $a$, can be defined in analogy to instantaneous
velocity as the average acceleration over an infinitesimally short time interval at
a given instant.

$a = \lim_{\Delta t \to 0} \frac{\Delta v}{\Delta t}$



\subsection{Deceleration}
is whenever the magnitude of the velocity is decreasing.

is when velocity and acceleration point in opposite directions.






\markStart[100]


\subsection{Motion at Constant Acceleration}

We now examine motion in a straight line when the magnitude of the acceleration is constant. In this case, the instantaneous and average accelerations are equal.


We use the definitions of average velocity and acceleration to derive a set of valuable equations that relate $x$, $v$, $a$, and $t$ when $a$ is constant, allowing us to determine any one of these variables if we know the others. We can then solve many interesting Problems.

First we choose the initial time in any discussion to be zero, and we call it $t_0$.
That is,  $t_1 = t_0 = 0$. (This is effectively starting a stopwatch at $t_0$ .)
We can then let $t_2 = t$ be the elapsed time.

The initial position $x_1$ and the initial velocity $v_1$  of an object will now be represented by $x_0$ and $v_0$ , since they represent $x$ and $v$ at $t = 0$.
At time $t$ the position and velocity will be called $x$ and $v$ (rather than $x_2$ and $v_2$).
The average velocity during the time interval $t - t_0$ will be

$\bar{v} = \frac{\Delta x}{\Delta t} = \frac{x - x_0}{t - t_0} = \frac{x - x_0}{t}$

since we chose $t_0 = 0$


$a = \frac{v - v_0}{t}$


The velocity of an object after any elapsed time t

$v = v_0 + at$

Calculating the 
Position x of an object after a time t

$\bar{v} = \frac{x - x_0}{t}$

becomes

$x = x_0 + \bar{v}t$


Because the velocity increases at a uniform rate $\bar{v}$ will be midway between the initial and final velocities

$\bar{v} = \frac{v_0 + v}{2}$


Combining the last three equations these become

$x = x_0 + \bar{v}t$

$x = x_0 + (\frac{v_0 + v}{2})t$

$x = x_0 + (\frac{v_0 + (v_0 + at)}{2})t$

$x = x_0 + v_0t + \frac{1}{2}at^2$



Three of the four most useful equations for motion at constant acceleration 

The velocity of an object after any elapsed time t

$v = v_0 + at$

Average velocity

$\bar{v} = \frac{v_0 + v}{2}$

Position x of an object after a time t

$x = x_0 + v_0t + \frac{1}{2}at^2$



Situations where time $t$ is not known

$x = x_0 + (\frac{v_0 + v}{2})t$


solve for $t$
$v = v_0 + at$

$t = \frac{v - v_0}{a}$

Substituting this into the previous equation we get 

$x = x_0 + (\frac{v_0 + v}{2})(\frac{v - v_0}{a}) = x_0 + \frac{v^2 - v_0^2}{2a}$

solve for $v^2$

$v^2 = v_0^2 + 2a(x - x_0)$

\end{tcolorbox}






Kinematic equations for constant acceleration 

$[a = \text{constant}]$

$v = v_0 + at$

$x = x_0 + v_0t + \frac{1}{2}at^2$

$v^2 = v_0^2 + 2a(x - x_0)$

$\bar{v} = \frac{v + v_0}{2}$


\markStart[100]
\subsection{Solving Problems}

1. Read and reread the whole problem carefully before trying to solve it.  

2. Decide what object (or objects) you are going to study, and for what time interval. You can often choose the initial time to be t = 0.  

3. Draw a diagram or picture of the situation, with coordinate axes wherever applicable. [You can place the origin of coordinates and the axes wherever you like to make your calculations easier. You also choose which direction is positive and which is negative. Usually we choose the x axis to the right as positive.]  

4. Write down what quantities are “known” or “given,” and then what you want to know. Consider quantities both at the beginning and at the end of the chosen time interval. You may need to “translate” language into physical terms, such as “starts from rest” means v0 = 0.  

5. Think about which principles of physics apply in this problem. Use common sense and your own experiences. Then plan an approach.  

6. Consider which equations (and/or definitions) relate the quantities involved. Before using them, be sure their range of validity includes your problem (for example, Eqs. 2–11 are valid only when the acceleration is constant). If you find an applicable  equation that involves only known quantities and one desired unknown, solve the equation algebraically for the unknown. Sometimes several sequential calculations, or a combination of equations, may be needed. It is often preferable to solve algebraically for the desired unknown before putting in numerical values.  

7. Carry out the calculation if it is a numerical problem. Keep one or two extra digits during the calculations, but round off the final answer(s) to the correct number of significant figures (Section 1–4).  

8. Think carefully about the result you obtain: Is it reasonable? Does it make sense according to your own intuition and experience? A good check is to do a rough estimate using only powers of 10, as discussed in Section 1–7. Often it is preferable to do a rough estimate at the start of a numerical problem because it can help you focus your attention on finding a path toward a solution.  

9. A very important aspect of doing problems is keeping track of units. An equals sign implies the units on each side must be the same, just as the numbers must. If the units do not balance, a mistake has been made. This can serve as a check on your solution (but it only tells you if you’re wrong, not if you’re right). Always use a consistent set of units.





\end{tcolorbox}


\subsection{Freely Falling Objects}

with upwards as positive

solving for t

$y = y_0 + v_0t + \frac{1}{2}at^2$

We rewrite our y equation just above in standard form,

$at^2 + bt + c = 0$

$0 = (y_0 - y) + v_0t + \frac{1}{2}at^2$

$\frac{1}{2}at^2 + v_0t + (y_0 - y) = 0$

$(\frac{1}{2}a)t^2 + (v_0)t + (y_0 - y) = 0$

Using the quadratic formula, we find as solutions

$t=\frac{-b\pm\sqrt{b^2-4ac}}{2a}$

$g = 9.80 m/s^2$

\markStart[25]

\subsection{Stuff about graphing}


\subsection{Right angled triangles}

$\sin\theta=Opp/Hyp$
	
$\cos\theta=Adj/Hyp$	
	
$\tan\theta=Opp/Adj$	


$\csc\theta=Hyp/Opp=1/\sin\theta$

$\sec\theta=Hyp/Adj=1/\cos\theta$

$\cot\theta=Adj/Opp=1/\tan\theta$

\end{tcolorbox}

\subsection{Resolving vectors}

$\mathbf{v}_x=r\cos{\theta}$

$\mathbf{v}_y=r\sin{\theta}$

$r=\left|\mathbf{v}\right|=\sqrt{x^2+y^2}$ (Magnitude)

$r=\left|\mathbf{v}\right|=\sqrt{x^2+y^2+z^2}$ (Magnitude for 3-space)

$\theta=\tan^{-1}{(\frac{\mathbf{v}_y}{\mathbf{v}_x})}$


\markStart[100]

$\theta = $\begin{tabularx}{\textwidth}{|c|c|}
    \hline
$180 - \theta$ & $\theta$ \\ \hline
$180 + \theta$ & $360 - \theta$ \\ \hline
    \end{tabularx}???

\end{tcolorbox}

    
    
    
    
    
    
    
    
\section{Kinematics in
Two Dimensions; Vectors}
    
    
\subsection{Solving Projectile Motion Problems}

General Kinematic Equations for Constant Acceleration
in Two Dimensions


\begin{tabularx}{\textwidth}{@{}X@{}X@{}}
x component (horizontal) & 
y component (vertical) \\
\hline
$v_x = v_{x0} + a_xt$ &
$v_y = v_{y0} + a_yt$ \\

$x = x_0 + v_{x0}t + \frac{1}{2}a_xt^2$ & 
$y = y_0 + v_{y0}t + \frac{1}{2}a_yt^2$ \\

$v_x^2 = v_{x0}^2 + 2a_x(x - x_0)$ & 
$v_y^2 = v_{y0}^2 + 2a_y(y - y_0)$ \\ \hline
\end{tabularx}



We can simplify Eqs to use for projectile motion because we can set

$a_x = 0$, $a_y = -g$


Kinematic Equations for Projectile Motion

(y positive upward; $a_x = 0, a_y = -g = 9.80 m/s^2$)




\begin{tabularx}{\textwidth}{@{}X@{}X@{}}
Horizontal Motion & 
Vertical Motion \\
($a_x = 0, v_x = \text{constant}$) &
($a_y = -g = \text{constant}$) \\
\hline
$v_x = v_{x0}$ &
$v_y = v_{y0} - gt$ \\

$x = x_0 + v_{x0}t $ & 
$y = y_0 + v_{y0}t - \frac{1}{2}gt^2$ \\

 & 
$v_y^2 = v_{y0}^2 - 2g(y - y_0)$ \\ \hline
\end{tabularx}




\subsection{Projectile Motion Is Parabolic}

 We now show that the path followed by any projectile is a parabola, if we can ignore air resistance and can assume that g is constant. 
 
 for simplicity we set $x_0 = y_0 = 0$
 
 
$x = v_{x0}t $

$y = v_{y0}t - \frac{1}{2}gt^2$ 
 
From the first equation, we have $t = \frac{x}{v_{x0}}$ , and we substitute this into the second one to obtain
 
$y = v_{y0}(\frac{x}{v_{x0}}) - \frac{1}{2}g(\frac{x}{v_{x0}})^2$ 

$y = (\frac{v_{y0}}{v_{x0}})x - \frac{1}{2}g(\frac{x^2}{v_{x0}^2})$ 

$y = (\frac{v_{y0}}{v_{x0}})x - (\frac{g}{2v_{x0}^2})x^2$

We see that $y$ as a function of $x$ has the form $y = Ax - Bx^2$, where $A$ and $B$ are constants for any specific projectile motion. This is the standard equation for a parabola.















\markStart[100]

Relative Velocity

We now consider how observations made in different frames of reference are related to each other.

For example, consider two trains approaching one another, each with a speed of 80 kmh with respect to the Earth.
Observers on the Earth beside the train tracks will measure 80 kmh for the speed of each of the trains.
Observers on either one of the trains (a different frame of reference) will measure a speed of 160 kmh for the train approaching them.

Similarly, when one car traveling 90 kmh passes a second car traveling in the same direction at 75 kmh, the first car has a speed relative to the second car of 90 kmh - 75 kmh = 15 kmh.


Use a diagram and a careful labeling process. Each velocity is labeled by two subscripts: the first refers to the object, the second to the reference frame in which it has this velocity.

$\vec{v}_{OR}$

Example, suppose a boat heads directly across a river

let

$\vec{v}_{BW}$ be the velocity of the Boat with respect to the Water.

$\vec{v}_{BS}$ be the velocity of the Boat with respect to the Shore, 

$\vec{v}_{WS}$ be the velocity of the Water with respect to the Shore 


$\vec{v}_{BS} = \vec{v}_{BW} + \vec{v}_{WS}$


By writing the subscripts using this convention, we see that the inner subscripts (the two W's) on the right-hand side of Eq. 3-7 are the same; also, the outer subscripts on the right of Eq. 3-7 (the B and the S) are the same as the two subscripts for the sum vector on the left, vBS . By following this convention (first subscript for the object, second for the reference frame), you can write down the correct equation relating velocities in different reference frames.


Equation 3-7 is valid in general and can be extended to three or more velocities. 

example, 

$\vec{v}_{FB}$ is the velocity of the fisherman relative to the boat

his velocity relative to the shore is $\vec{v}_{FS} = \vec{v}_{FB} + \vec{v}_{BW} + \vec{v}_{WS}$

The equations involving relative velocity will be correct when

there is no vector subtraction 

adjacent inner subscripts are identical

and when the outermost ones correspond exactly to the two on the velocity on the left of the equation.

It is often useful to remember that for any two objects or reference frames, A and B, the velocity of A relative to B has the same magnitude, but opposite direction, as the velocity of B relative to A:

$\vec{v}_{AB} = -\vec{v}_{BA}$


\end{tcolorbox}













\markStart[100]



\section{Work and Energy}

\subsection{Kinetic Energy, and the Work-Energy Principle}

To obtain a quantitative definition for kinetic energy, let us consider a simple rigid object of mass $m$ (treated as a particle) that is moving in a straight line with an initial speed $v_1$. To accelerate it uniformly to a speed $v_2$ a constant net force $F_{net}$ is exerted on it parallel to its motion over a displacement $d$, Fig. 6–7. Then the net work done on the object is $W_{net} = F_{net}d$. We apply Newton’s second law, $F_{net} = ma$ and use Eq. 2–11c $(v_2^2 = v_1^2 + 2ad)$ which we rewrite as
$a = \frac{v_2^2 - v_1^2}{2d}$

where $v_1$ is the initial speed and $v_2$ is the final speed. Substituting this into $F_{net} = ma$, we determine the work done:

$W_{net} = F_{net}d = mad = m(\frac{v_2^2 - v_1^2}{2d})d = m(\frac{v_2^2 - v_1^2}{2})$

or

$W_{net} = \frac{1}{2}mv_2^2 - \frac{1}{2}mv_1^2$


We define the quantity
to be the translational kinetic energy (KE) of the object:

$KE = \frac{1}{2}mv^2$

so 

$W_{net} = {KE}_2 - {KE}_1$

or 

work-energy principle

$W_{net} = \Delta KE = \frac{1}{2}mv_2^2 - \frac{1}{2}mv_1^2$

It can be stated in words:

The net work done on an object is equal to the change in the object’s
kinetic energy.


Thus, the work-energy principle is valid only if W is the net work done on the object—that is, the work done by all forces acting on the object.





\markEnd

\markStart[100]










Potential Energy Defined in General

In general, the change in potential energy
associated with a particular force is equal to the negative of the work done by
that force when the object is moved from one point to a second point (as in Eq. 6–7b
for gravity). Alternatively, we can define the change in potential energy as the
work required of an external force to move the object without acceleration between
the two points


















FS =–kx.

where k is a constant, called the spring stiffness constant (or simply spring constant),
and is a measure of the stiffness of the particular spring.

spring either
stretched or compressed an amount x from its natural (unstretched) length


spring equation and
also as Hooke’s law, and is accurate for springs as long as x is not too great.





elastic potential energy




\markEnd








\subsection{Phys1006 formula sheet}





DATA

\markStart[50]
\begin{tabularx}{\textwidth}{@{}X@{}r@{}l@{}}
Avogadro's number & $\mathrm{N}_{\mathrm{A}}$ & $=6.02 \times 10^{23} \mathrm{mol}^{-1}$ \\
Planck's constant & $\mathrm{h}$ & $=6.626 \times 10^{-34} \mathrm{J} \mathrm{s}$ \\
Stephan-Boltzmann's constant & $\sigma$ & $=5.67 \times 10^{-8} \mathrm{W} \mathrm{m}^{-2} \mathrm{K}^{-4}$ \\
Boltzmann's constant & $\mathrm{k}_{\mathrm{B}}$ & $=1.38 \times 10^{-23} \mathrm{J} \mathrm{K}^{-1}$ \\



Gravitational constant & $\mathrm{G}$ & $=6.67 \times 10^{-11} \mathrm{N} \cdot \mathrm{m}^{2} \cdot \mathrm{kg}^{-2}$ \\
Radius of the Earth & $\mathrm{R}_{\mathrm{E}}$ & $=6.38 \times 10^{6} \mathrm{m}$ \\
Mass of the Earth & $\mathrm{M}_{\mathrm{E}}$ & $=5.98 \times 10^{24} \mathrm{kg}$ \\
Gas constant & $\mathrm{R}$ & $=8.314 \mathrm{J} / \mathrm{mol} . \mathrm{K}$ \\
Permittivity of free space & $\varepsilon_{\mathrm{o}}$ & $=8.85 \times 10^{-12} \mathrm{C}^{2} / \mathrm{N} \cdot \mathrm{m}^{2}$ \\
Permeability of free space & $\mathrm{\mu o}$ & $=4 \pi \times 10^{-7} \mathrm{T} . \mathrm{m} / \mathrm{A}$ \\
Coulomb constant & $\mathrm{k}$ & $=9.0 \times 10^{9} \mathrm{N} \mathrm{m}^{2} / \mathrm{C}^{2}$ \\

\end{tabularx}

\begin{tabularx}{\textwidth}{@{}X@{}r@{}l@{}}

acceleration due to gravity & $\mathrm{g}$ & $=9.80 \mathrm{m} \mathrm{s}^{-2}$ \\
index of refraction of air (STP) & $\mathrm{n}$ & $=1.0003$ \\
speed of light (in vacuum) & $\mathrm{c}$ & $=3 \times 10^{8} \mathrm{ms}^{-1}$ \\
speed of sound (at $\left.0^{\circ} \mathrm{C}\right)$ & $\mathrm{v}$ & $=331 \mathrm{ms}^{-1}$ \\
Pi & $\pi$ & $=3.1416$ \\
Volume & $1 \text {litre}$ & $=1000 \mathrm{cm}^{3}$ \\
density of water & $\rho$ & $=1000 \mathrm{kg} \mathrm{m}^{-3}$ \\
density of air & & $=1.29 \mathrm{kg} \mathrm{m}^{-3}$ \\
atmospheric pressure & $1 \mathrm{atm}$ & $=1.013 \times 10^{5} \mathrm{Pa}$ \\
volume of air at STP & & $=22.4 \mathrm{L}$ \\
coefficient of thermal conductivity for brick & & $=0.84 \mathrm{J} \mathrm{m}^{-1} \mathrm{s}^{-1} \mathrm{K}^{-1}$ \\
coefficient of thermal conductivity for glass & & $=0.84 \mathrm{J} \mathrm{m}^{-1} \mathrm{s}^{-1} \mathrm{K}^{-1}$ \\
zero Kelvin & & $=-273^{\circ} \mathrm{C}$ \\
freezing point of water & & $=0^{\circ} \mathrm{C}=32^{\circ} \mathrm{F}$ \\
boiling point of water & & $=100^{\circ} \mathrm{C}=212^{\circ} \mathrm{F}$ \\

\end{tabularx}

\begin{tabularx}{\textwidth}{@{}X@{}r@{}l@{}}

specific heat of water & $\mathrm{c}_{\text {water }}$ & $=4186 \mathrm{J} \mathrm{kg}^{-1} \mathrm{C}^{\circ-1}$ \\
specific heat of ice & $\quad \mathrm{c}_{\mathrm{ice}}$ & $=2100 \mathrm{J} \mathrm{kg}^{-1} \mathrm{C}^{\circ-1}$ \\
specific heat of iron & $\quad \mathrm{c}_{\text {iron }}$ & $=450 \mathrm{J} \mathrm{kg}^{-1} \mathrm{C}^{\circ-1}$ \\
specific heat of copper & $\mathrm{c}_{\mathrm{cu}}$ & $=390 \mathrm{J} \mathrm{kg}^{-1} \mathrm{C}^{\circ-1}$ \\
specific heat of aluminum & $\mathrm{c}_{\mathrm{Al}}$ & $=900 \mathrm{J} \mathrm{kg}^{-1} \mathrm{C}^{\circ-1}$ \\

\end{tabularx}

\begin{tabularx}{\textwidth}{@{}X@{}r@{}l@{}}

latent heat of vaporization of water & & $=2.26 \times 10^{6} \mathrm{J} \mathrm{kg}^{-1}$ \\
latent heat of fusion of ice & & $=3.33 \times 10^{5} \mathrm{J} \mathrm{kg}^{-1}$ \\
coefficient of volume expansion of petrol & & $=950 \times 10^{-6} \mathrm{C}^{\circ-1}$ \\
coefficient of linear expansion of steel/iron & & $=12 \times 10^{-6} \mathrm{C}^{\circ-1}$ \\
coefficient of linear expansion of brass & & $=19 \times 10^{-6} \mathrm{C}^{\circ-1}$ \\

\end{tabularx}

\begin{tabularx}{\textwidth}{@{}X@{}r@{}l@{}}

charge on an electron & & $=1.6 \times 10^{-19} \mathrm{C}$ \\
$1 \mathrm{eV}$ & & $=1.6 \times 10^{-19} \mathrm{J}$ \\

\end{tabularx}


\markEnd



FORMULA SHEET

MODULE 1: FUNDAMENTALS
\markStart[25]

$\sin \theta=\frac{\text { opposite }}{\text { hypotenuse }}$

$\cos \theta=\frac{\text { adjacent }}{\text { hypotenuse }}$

$\tan \theta=\frac{\text { opposite }}{\text { adjacent }}$

\markEnd

\markStart[100]


Area of a circle $=\pi r^{2}$

Circumference of a circle $=2 \pi r$

Surface area of a sphere $=4 \pi r^{2} \quad$

Volume of a sphere $=\frac{4}{3} \pi r^{3}$

Volume of a cylinder $=\pi r^{2} h \quad$

Density $(\rho)=\frac{\operatorname{mass}(m)}{\text {volume }(V)}$

$v_{\text {average}}=\frac{v_{i}+v_{f}}{2}$

$\left(x_{f}-x_{i}\right)=$displacement$/$distance

$v_{f}=v_{i}+a t$

$v_{f}^{2}=v_{i}^{2}+2 a\left(x_{f}-x_{i}\right)$

$\left(x_{f}-x_{i}\right)=v_{i} t+\frac{1}{2} a t^{2}$

$ \mathrm{R}=\frac{\mathrm{v}_{\mathrm{i}}^{2} \sin 2 \theta}{g}, \mathrm{t}=\frac{\mathrm{v} \sin \theta}{g}$

$F=m a \quad \quad F_{\text {friction}}=\mu N \quad W=m g \quad N=m g \cos \theta$

$\mu_{S}=\tan \theta_{c} \quad F_{x}=F \cos \theta, \quad F_{y}=F \sin \theta, F=\sqrt{\left(F_{x}\right)^{2}+\left(F_{y}\right)^{2}}, \tan \theta=\frac{F_{y}}{F_{x}}$

$\text { Work Done }=F d \cos \theta$

$ \text { Kinetic Energy }=\frac{1}{2} m v^{2}$

Potential Energy $=m g h$

Work Done $=\Delta$ Kinetic Energy

Total Mechanical Energy $=$ Kinetic Energy $+$ Potential Energy

$P=\frac{\text {Work}}{\text {time}}=\frac{W}{t}=F v$

$\text { Normal force in an elevator } \mathrm{N}=\mathrm{mg} \pm \mathrm{ma}$

\end{tcolorbox}

\markStart[25]

\subsubsection{Propagation of Uncertainties} 

For Additions or Subtractions of measured values

$\mathrm{C}=\mathrm{A} \pm \mathrm{B} \quad \Delta \mathrm{C}=\Delta \mathrm{A} \pm \Delta \mathrm{B}$


For Multiplications and Divisions of measured values

$x=\frac{k^{a} t^{b}}{m^{c} n^{d}} \quad \frac{\Delta x}{x}=a \frac{\Delta k}{k}+b \frac{\Delta t}{t}+c \frac{\Delta m}{m}+d \frac{\Delta n}{n}$


\end{tcolorbox}


MODULE 2:

Data
\markStart[100]

\begin{tabularx}{\textwidth}{@{}X@{}r@{}l@{}}

Charge on electron & & $=1.6 \times 10^{-19} \mathrm{C} \quad$ \\

Mass of electron & & $=9.11 \times 10^{-31} \mathrm{kg}$ \\

Charge on proton & & $=1.6 \times 10^{-19} \mathrm{C} \quad$ \\

Mass of proton & & $=1.67 \times 10^{-27} \mathrm{kg}$ \\

Coulomb constant & $\mathrm{k}$ & $=9.0 \times 10^{9} \mathrm{N} . \mathrm{m}^{2} / \mathrm{C}^{2}$ \\

$1 \mathrm{eV}$ & & $=1.6 \times 10^{-19} \mathrm{J}$ \\

Permittivity of free space & $\varepsilon \mathrm{O}$ & $=8.85 \times 10^{-12} \mathrm{C}^{2} / \mathrm{N} . \mathrm{m}^{2}$ \\

Permeability of free space & $\mu \mathrm{o}$ & $=4 \pi \times 10^{-7} \mathrm{T} . \mathrm{m} / \mathrm{A}$ \\

Resistivity of nichrome, & $\rho$ & $=100 \times 10^{-8} \Omega . \mathrm{m}$ \\

Resistivity of copper, & $\rho$ & $=1.68 \times 10^{-8} \Omega . \mathrm{m}$ \\

speed of light (in vacuum) & $c$ & $=3$ x $10^{8} \mathrm{ms}^{-1}$ \\

\end{tabularx}

\markEnd

\markStart[100]


Formula Sheet

$\mathrm{F}=\frac{k q_{1} q_{2}}{r^{2}} \quad \mathrm{E}_{\text {isolated charge }}=\frac{F}{q}=\frac{k q_{1}}{r^{2}} \quad \mathrm{V}=\mathrm{Ed}(\text { uniform } \mathbf{E})$

$\mathrm{V}=\mathrm{IR}$

$\mathrm{R}=\frac{\rho \mathrm{L}}{\mathrm{A}} \quad I=Q / t$


$\begin{array}{lll}

\mathrm{V}=\mathrm{IR} & \mathrm{R}=\frac{\rho \mathrm{L}}{\mathrm{A}} & I=Q / t \\

\mathrm{V}=\mathrm{IR} & \mathrm{W}=\mathrm{qV} & \mathrm{P}=\mathrm{VI}=\mathrm{I}^{2} R=\frac{V^{2}}{R} \\

\mathrm{C}=\frac{\varepsilon_{0} A}{\mathrm{d}}=\frac{k \varepsilon_{0} A}{\mathrm{d}} & Q=C V & \mathrm{U}=\frac{1}{2} C V^{2} \\

\mathrm{R}_{\text {series }}=\sum_{i} R_{i}=R+R_{2}+R_{3} & \frac{1}{\mathrm{R}_{\text {parallel }}}=\sum_{i} \frac{1}{R_{i}}=\frac{1}{R_{1}}+\frac{1}{R_{2}}+\frac{1}{R_{3}} \\

\mathrm{P}=\mathrm{VI}=\mathrm{I}^{2} R=\frac{V^{2}}{R} & \mathrm{R}_{\mathrm{T}}=\mathrm{R}_{\mathrm{o}}(1+\alpha \Delta T) \\

\mathrm{P}_{\max }=\mathrm{I}_{\mathrm{o}} V_{o} & \mathrm{P}_{\text {average }}=\mathrm{I}_{\mathrm{rms}} V_{r m s}=\mathrm{I}_{\mathrm{o}} V_{o} / 2 \\

\mathrm{F}=\mathrm{qvBsin} \theta & \mathrm{F}=\mathrm{IlBsin} \theta

\end{array}$

$\begin{array}{ll}

\mathrm{F}=\mathrm{qvBsin} \theta \quad & \mathrm{F}=\mathrm{IlBsin} \theta \\

\phi_{\mathrm{B}}=A B \cos \theta & \varepsilon=\frac{-N \Delta \Phi_{B}}{\Delta t} \quad \varepsilon=\mathrm{Blvsin} \theta \\

B=\frac{\mu_{o} I}{2 \pi r} & \frac{V_{S}}{V_{P}}=\frac{N_{S}}{N_{p}}=\frac{I_{P}}{I_{S}}

\end{array}$

\end{tcolorbox}



MODULE 3: WAVES and SOUND

Also see Fundamental Principles formulae
\markStart[100]

$F=k x \quad[\text {spring}]$

$\left.P E=1 / 2 \mathrm{kx}^{2} \text { [spring }\right]$

$\mathrm{KE}=1 / 2 \mathrm{mv}^{2}$

$\text { Total } E=\mathrm{KE}+\mathrm{PE}$

$\left.T=2 \pi(m / k)^{1 / 2} \text { [mass-spring }\right]$

$f=1 / T$

$T=2 \pi(L / g)^{I / 2}[\text { pendulum }]$

$\omega=2 \pi \mathrm{f}$

$\mathrm{v}=\mathrm{f} \lambda$

$v=(331+0.6 T) \mathrm{m} / \mathrm{s}$

$x=A \cos (\omega t)$

$v=-A \omega \sin (\omega t) \quad a=-A \omega^{2} \cos (\omega t)$

$\mathrm{V}_{\max }=A \omega$

$\mathrm{v}=\sqrt{\frac{\mathrm{k}}{\mathrm{m}}\left(A^{2}-x^{2}\right)} \quad a_{\max }=\omega^{2} A$

Speed of wave in a string/wire, $\mathrm{V}=\sqrt{\frac{\mathrm{F}_{\mathrm{T}}}{\mathrm{m} / \mathrm{L}}}$


Speed of wave in a solid, $v=\sqrt{\frac{E}{\rho}}$

Speed of wave in a liquid/gas, $\mathrm{v}=\sqrt{\frac{\mathrm{B}}{\rho}}$

Harmonics: strings/open pipes, $\quad \mathrm{f}_{\mathrm{n}}=\mathrm{n}\left(\frac{v}{2 L}\right) \quad$ Closed pipes, $\mathrm{f}_{\mathrm{n}}=\mathrm{n}\left(\frac{v}{4 L}\right)$

Law of reflection, $\theta_{i}=\theta_{r}$

Law of refraction, $v_{1} \sin \theta_{1}=v_{2} \sin \theta_{2}$

Intensity level $\beta(\mathrm{dB})=10 \log \left(\frac{\mathrm{I}}{\mathrm{I}_{0}}\right)$

where $I_{0}=1.0 \times 10^{-12} W / m^{2}$

$\beta_{2}-\beta_{1}=\Delta \beta=\log \left(\frac{I_{2}}{I_{1}}\right) \quad \mathrm{I}=\text { Power } / 4 \pi r^{2}$


\end{tcolorbox}



MODULE 4: THERMAL PHYSICS
\markStart[100]

$T\left({ }^{o} C\right)=\frac{5}{9}[T(F)-32]$

$\left.\Delta L=\alpha L_{o} \Delta T \text { linear thermal expansion }\right]$

$\left.\Delta V=\beta V_{O} \Delta T \text { [volume thermal expansion }\right]$

$Q=m c \Delta T[\text {Specific heat}]$

$Q=m L[\text {Latent heat}]$

Rate of heat flow by conduction $\frac{\Delta \mathrm{Q}}{\Delta \mathrm{t}}=\mathrm{kA} \frac{\Delta \mathrm{T}}{\mathrm{L}}$

Thermal insulation $\mathrm{R}=\frac{\mathrm{L}}{\mathrm{k}}$

Net flow rate of heat radiation $P=\sigma A e\left(T_{\text {body}}^{4}-T_{\text {enviromment}}^{4}\right)$

$\frac{P_{1} V_{1}}{T_{1}}=\frac{P_{2} V_{2}}{T_{2}}$

$P V=n R T \quad \text { (Ideal gas law) })$

$N=n N_{A} \quad(\mathrm{N}=\text { number of molcules in a gas smaple })$

$P V=N k_{B} T \quad\left(\mathrm{k}_{\mathrm{B}}=\frac{\mathrm{R}}{\mathrm{N}_{\mathrm{A}}}\right)$

$\overline{\mathrm{KE}}=\frac{3}{2} k_{B} T$

(Average KE of a molecule)

$\mathrm{v}_{\mathrm{rms}}=\sqrt{\frac{3 k_{B} T}{m}}(r m s \text { speed of moecule }) \quad$ where $\mathrm{m}$ is the mass of a molecule


\end{tcolorbox}


MODULE 5: OPTICS
\markStart[100]

Index of refraction $n=c / \nu \quad$ Snell's Law, $n_{1} \sin \theta_{1}=n_{2} \sin \theta_{2}$

Mirror/Lens formula $\quad(1 / f)=\left(1 / d_{O}\right)+\left(1 / d_{i}\right)$

Magnification $M=-\left(d_{i} / d_{O}\right)=\left(-h_{i} / h_{O}\right)$

Lens makers' formula $(1 / f)=(n-1)\left(1 / R_{1}+1 / R_{2}\right)$

$P=\frac{1}{f(m)}$

Magnifying glass $M_{\text {infinity}}=N / f \quad M_{N}=1+N / f$

Microscope $M=M_{O} M_{e}=(N \times L) / f_{O} f_{e}$

Rayleigh criterion for resolution $\quad \theta_{\min }=1.22 \lambda / \mathrm{D}$

Resolving power of a microscope $S=f(1.22 \lambda / D)$

Brewster's Law $\quad \tan \theta_{\mathrm{p}}=\mathrm{n}_{2} / \mathrm{n}_{1}$

Malus' Law: : : : : : Lawlus' Law:: : : :

$\mathrm{I}=(1 / 2) \mathrm{I}_{\mathrm{O}} \cos ^{2} \theta$



\end{tcolorbox}

MODULE 6: NUCLEAR RADIATION 

\markStart[100]

\begin{tabularx}{\textwidth}{@{}X@{}r@{}l@{}}

charge on an electron & & $=1.6 \times 10^{-19} \mathrm{C}$ \\

$1 \mathrm{eV}$ & & $=1.6 \times 10^{-19} \mathrm{J}$ \\

mass of neutron & & $=1.008665 \mathrm{u}$ \\

mass of hydrogen atom & & $=1.007825 \mathrm{u}$ \\

mass of ${ }^{4} \mathrm{He}$ atom & & $=4.002602 \mathrm{u}$ \\

atomic mass unit & $1 \mathrm{u}$ & $=1.66 \times 10^{-27} \mathrm{kg}, =931.5 \mathrm{MeV} / \mathrm{c}^{2}$ \\

mass of electron & & $=9.11 \times 10^{-31} \mathrm{kg}$ \\

mass of proton & & $=1.67 \mathrm{x} 10^{-27} \mathrm{kg}$ \\

\end{tabularx}



\markEnd

\markStart[100]




$f \lambda=c \quad E=h f$

$P=\sigma A e T^{4} \quad \lambda_{M A X} T=2.89 \times 10^{-3} \mathrm{mK}$

$h f=W_{0}+K E_{M A X} \quad K E_{M A X}=e V_{\text {STOPPING}}$

$E_{n}=\frac{-13.6}{n^{2}} e V \quad E=13.6\left(\frac{1}{n_{f}^{2}}-\frac{1}{n_{i}^{2}}\right) e V$

$N=N_{0} \exp (-\lambda t) \quad$ (number of atoms) $\quad A=\lambda N \quad$ (activity)

$A=A_{0} \exp (-\lambda t) \quad T_{1 / 2}=\frac{\ln (2)}{\lambda}$

Binding Energy $=\left(Z m_{p}+N m_{n}-M_{A}\right) \times 931.5 \mathrm{MeV} / u$

Absorbed dose: $1 G y=1 J / k g$ dose $=$ dose rate x time









Equivalent dose $=$ absorbed dose $\mathrm{x}$ w $\mathrm{R}$

Effective dose $=$ Equivalent dose $\mathrm{x}$ wr











\end{tcolorbox}




$\begin{array}{|l|c|l|c|c|}

\end{array}$


%https://tex.stackexchange.com/a/89932

\subsection{Named units derived from SI base units}
\markStart[100]
%Name & Symbol & Quantity & Equivalents & SI base unit \\
$
\begin{tabularx}{\textwidth}{@{}|@{}l@{}|@{}c@{}|@{}X@{}|@{}Y@{}|@{}c@{}|@{}}
%Name & Symbol & Quantity & Equivalents & SI base unit \\

% \hline
%Name & Sym & Quantity & Equivalents & SI base unit \\
% \hline
%\multicolumn{5}{|@{}l@{}|}{Equivalents} \\
 \hline
hertz & Hz & frequency & 1/s & s1 \\
 \hline
radian & rad & angle & m/m & 1 \\
 \hline
steradian & sr & solid angle & m2/m2 & 1 \\
 \hline
newton & N & force, weight & kgm/s2 & kgms2 \\
 \hline
pascal & Pa & pressure, stress & N/m2 & kgm1s2 \\
 \hline
joule & J & energy, work, heat & mN, CV, Ws & kgm2s2 \\
 \hline
watt & W & power, radiant flux & J/s, VA & kgm2s3 \\
 \hline
coulomb & C & electric charge or quantity of electricity & sA, FV & sA \\
 \hline
volt & V & voltage, electrical potential difference, electromotive force & W/A, J/C & kgm2s3A1 \\
 \hline
farad & F & electrical capacitance & C/V, s/ & kg1m2s4A2 \\
 \hline
ohm &  & electrical resistance, impedance, reactance & 1/S, V/A & kgm2s3A2 \\
 \hline
siemens & S & electrical conductance & 1/, A/V & kg1m2s3A2 \\
 \hline
weber & Wb & magnetic flux & J/A, Tm2,Vs & kgm2s2A1 \\
 \hline
tesla & T & magnetic induction, magnetic flux density & Vs/m2, Wb/m2, N/(Am) & kgs2A1 \\
 \hline
henry & H & electrical inductance & Vs/A, s, Wb/A & kgm2s2A2 \\
 \hline
degree Celsius & C & temperature relative to 273.15 K & K & K \\
 \hline
lumen & lm & luminous flux & cdsr & cd \\
 \hline
lux & lx & illuminance & lm/m2 & cdm2 \\
 \hline
becquerel & Bq & radioactivity (decays per unit time) & 1/s & s1 \\
 \hline
gray & Gy & absorbed dose (of ionizing radiation) & J/kg & m2s2 \\
 \hline
sievert & Sv & equivalent dose (of ionizing radiation) & J/kg & m2s2 \\
 \hline
katal & kat & catalytic activity & mol/s & s1mol \\
 \hline

\end{tabularx}
$




\end{tcolorbox}


\subsection{Kinematic SI derived units}
\markStart[100]

$
\begin{tabularx}{\textwidth}{@{}|@{}X@{}|@{}c@{}|@{}X@{}|@{}l@{}|}
% \hline
%Name & Symbol & Quantity & Expression in terms of SI base units \\
 \hline
metre per second & m/s & speed, velocity & ms1 \\
 \hline
metre per second squared & m/s2 & acceleration & ms2 \\
 \hline
metre per second cubed & m/s3 & jerk, jolt & ms3 \\
 \hline
metre per second to the fourth & m/s4 & snap, jounce & ms4 \\
 \hline
radian per second & rad/s & angular velocity & s1 \\
 \hline
radian per second squared & rad/s2 & angular acceleration & s2 \\
 \hline
hertz per second & Hz/s & frequency drift & s2 \\
 \hline
cubic metre per second & m3/s & volumetric flow & m3s1 \\
 \hline
\end{tabularx}
$
\end{tcolorbox}


\subsection{Mechanical SI derived units}
\markStart[100]

$
\begin{tabularx}{\textwidth}{@{}|@{}X@{}|@{}c@{}|@{}X@{}|@{}l@{}|}
% \hline
%Name & Symbol & Quantity & Expression in terms of SI base units \\
 \hline
square metre & m2 & area & m2 \\
 \hline
cubic metre & m3 & volume & m3 \\
 \hline
newton second & Ns & momentum, impulse & mkgs1 \\
 \hline
newton metre second & Nms & angular momentum & m2kgs1 \\
 \hline
newton metre & Nm = J/rad & torque, moment of force & m2kgs2 \\
 \hline
newton per second & N/s & yank & mkgs3 \\
 \hline
reciprocal metre & m1 & wavenumber, optical power, curvature, spatial frequency & m1 \\
 \hline
kilogram per square metre & kg/m2 & area density & m2kg \\
 \hline
kilogram per cubic metre & kg/m3 & density, mass density & m3kg \\
 \hline
cubic metre per kilogram & m3/kg & specific volume & m3kg1 \\
 \hline
joule second & Js & action & m2kgs1 \\
 \hline
joule per kilogram & J/kg & specific energy & m2s2 \\
 \hline
joule per cubic metre & J/m3 & energy density & m1kgs2 \\
 \hline
newton per metre & N/m = J/m2 & surface tension, stiffness & kgs2 \\
 \hline
watt per square metre & W/m2 & heat flux density, irradiance & kgs3 \\
 \hline
square metre per second & m2/s & kinematic viscosity, thermal diffusivity, diffusion coefficient & m2s1 \\
 \hline
pascal second & Pas = Ns/m2 & dynamic viscosity & m1kgs1 \\
 \hline
kilogram per metre & kg/m & linear mass density & m1kg \\
 \hline
kilogram per second & kg/s & mass flow rate & kgs1 \\
 \hline
watt per steradian square metre & W/(srm2) & radiance & kgs3 \\
 \hline
watt per steradian cubic metre & W/(srm3) & spectral radiance & m1kgs3 \\
 \hline
watt per metre & W/m & spectral power & mkgs3 \\
 \hline
gray per second & Gy/s & absorbed dose rate & m2s3 \\
 \hline
metre per cubic metre & m/m3 & fuel efficiency & m2 \\
 \hline
watt per cubic metre & W/m3 & spectral irradiance, power density & m1kgs3 \\
 \hline
joule per square metre second & J/(m2s) & energy flux density & kgs3 \\
 \hline
reciprocal pascal & Pa1 & compressibility & mkg1s2 \\
 \hline
joule per square metre & J/m2 & radiant exposure & kgs2 \\
 \hline
kilogram square metre & kgm2 & moment of inertia & m2kg \\
 \hline
newton metre second per kilogram & Nms/kg & specific angular momentum & m2s1 \\
 \hline
watt per steradian & W/sr & radiant intensity & m2kgs3 \\
 \hline
watt per steradian metre & W/(srm) & spectral intensity & mkgs3 \\
 \hline
\end{tabularx}
$
\end{tcolorbox}


\subsection{Molar SI derived units}
\markStart[100]

$
\begin{tabularx}{\textwidth}{@{}|@{}X@{}|@{}c@{}|@{}X@{}|@{}l@{}|}
% \hline
%Name & Symbol & Quantity & Expression in terms of SI base units \\
 \hline
mole per cubic metre & mol/m3 & molarity, amount of substance concentration & m3mol \\
 \hline
cubic metre per mole & m3/mol & molar volume & m3mol1 \\
 \hline
joule per kelvin mole & J/(Kmol) & molar heat capacity, molar entropy & m2kgs2K1mol1 \\
 \hline
joule per mole & J/mol & molar energy & m2kgs2mol1 \\
 \hline
siemens square metre per mole & Sm2/mol & molar conductivity & kg1s3A2mol1 \\
 \hline
mole per kilogram & mol/kg & molality & kg1mol \\
 \hline
kilogram per mole & kg/mol & molar mass & kgmol1 \\
 \hline
cubic metre per mole second & m3/(mols) & catalytic efficiency & m3s1mol1 \\
 \hline
\end{tabularx}
$
\end{tcolorbox}


\subsection{Electromagnetic SI derived units }
\markStart[100]

$
\begin{tabularx}{\textwidth}{@{}|@{}X@{}|@{}c@{}|@{}X@{}|@{}l@{}|}
% \hline
%Name & Symbol & Quantity & Expression in terms of SI base units \\
 \hline
coulomb per square metre & C/m2 & electric displacement field, polarization density & m2sA \\
 \hline
coulomb per cubic metre & C/m3 & electric charge density & m3sA \\
 \hline
ampere per square metre & A/m2 & electric current density & m2A \\
 \hline
siemens per metre & S/m & electrical conductivity & m3kg1s3A2 \\
 \hline
farad per metre & F/m & permittivity & m3kg1s4A2 \\
 \hline
henry per metre & H/m & magnetic permeability & mkgs2A2 \\
 \hline
volt per metre & V/m & electric field strength & mkgs3A1 \\
 \hline
ampere per metre & A/m & magnetization, magnetic field strength & m1A \\
 \hline
coulomb per kilogram & C/kg & exposure (X and gamma rays) & kg1sA \\
 \hline
ohm metre & m & resistivity & m3kgs3A2 \\
 \hline
coulomb per metre & C/m & linear charge density & m1sA \\
 \hline
joule per tesla & J/T & magnetic dipole moment & m2A \\
 \hline
square metre per volt second & m2/(Vs) & electron mobility & kg1s2A \\
 \hline
reciprocal henry & H1 & magnetic reluctance & m2kg1s2A2 \\
 \hline
weber per metre & Wb/m & magnetic vector potential & mkgs2A1 \\
 \hline
weber metre & Wbm & magnetic moment & m3kgs2A1 \\
 \hline
tesla metre & Tm & magnetic rigidity & mkgs2A1 \\
 \hline
ampere radian & Arad & magnetomotive force & A \\
 \hline
metre per henry & m/H & magnetic susceptibility & m1kg1s2A2 \\
 \hline
\end{tabularx}
$
\end{tcolorbox}


\subsection{Photometric SI derived units}
\markStart[100]

$
\begin{tabularx}{\textwidth}{@{}|@{}X@{}|@{}c@{}|@{}X@{}|@{}l@{}|}
% \hline
%Name & Symbol & Quantity & Expression in terms of SI base units \\
 \hline
lumen second & lms & luminous energy & scd \\
 \hline
lux second & lxs & luminous exposure & m2scd \\
 \hline
candela per square metre & cd/m2 & luminance & m2cd \\
 \hline
lumen per watt & lm/W & luminous efficacy & m2kg1s3cd \\
 \hline
\end{tabularx}
$
\end{tcolorbox}


\subsection{Thermodynamic SI derived units}
\markStart[100]

$
\begin{tabularx}{\textwidth}{@{}|@{}X@{}|@{}c@{}|@{}X@{}|@{}l@{}|}
% \hline
%Name & Symbol & Quantity & Expression in terms of SI base units \\
 \hline
joule per kelvin & J/K & heat capacity, entropy & m2kgs2K1 \\
 \hline
joule per kilogram kelvin & J/(Kkg) & specific heat capacity, specific entropy & m2s2K1 \\
 \hline
watt per metre kelvin & W/(mK) & thermal conductivity & mkgs3K1 \\
 \hline
kelvin per watt & K/W & thermal resistance & m2kg1s3K \\
 \hline
reciprocal kelvin & K1 & thermal expansion coefficient & K1 \\
 \hline
kelvin per metre & K/m & temperature gradient & m1K \\
 \hline
\end{tabularx}
$
\end{tcolorbox}





\section{Latex tools}

https://www.tablesgenerator.com/latex\_tables

http://w2.syronex.com/jmr/latex-symbols-converter




\iftoggle{editing}{% editing
}{% finaloutput
  \end{multicols}
}







\end{document}
