\documentclass{extarticle}
\usepackage[utf8]{inputenc}

%https://tex.stackexchange.com/a/5896
\usepackage{etoolbox}
\newtoggle{editing}
\toggletrue{editing}
%\togglefalse{editing}
%example
\iftoggle{editing}{%
  % editing
}{%
  % finaloutput
}


\iftoggle{editing}{%
  % editing
  \usepackage[12pt]{extsizes}
  \usepackage{geometry}
  \geometry{a6paper, portrait, margin=0.5cm, bottom=1.5cm}
  %, right=5cm
}{%
  % finaloutput
  \usepackage[8pt]{extsizes}
  \usepackage{geometry}
  \geometry{a4paper, portrait, margin=0.5cm, bottom=1.5cm}
  %, right=5cm
}



\usepackage{fancyhdr}
\pagestyle{fancy}
\fancyfoot[L]{\jobname\hspace{1cm}  Page \thepage}
\fancyfoot[C]{}
\fancyfoot[R]{}

%\documentclass[20pt]{extarticle}
%\documentclass[a5paper,11pt]{article}
%\documentclass[6pt]{extarticle}
%\documentclass[14pt]{extarticle}
%\documentclass[a3paper,11pt]{article}


%\usepackage[a3paper,margin=0.1in]{geometry}
%\usepackage[a3paper]{geometry}



\usepackage{verbatim}
\usepackage{mathptmx}
\usepackage{amsfonts}
\usepackage{amssymb}
\usepackage{gensymb}
%\usepackage{extsizes}

\usepackage{multicol}
\usepackage{tabu}




\usepackage[dvipsnames]{xcolor}

\usepackage[most]{tcolorbox}
\tcbuselibrary{breakable}

%\usepackage[subtle]{savetrees}
%\usepackage[moderate]{savetrees}
\usepackage[extreme]{savetrees}

\raggedright
\raggedbottom

\begin{document} 

\iftoggle{editing}{%
  % editing
}{%
  % finaloutput
  \begin{multicols}{4}
}
%\chapter{First Chapter}


%\smallskip
%\medskip
%\bigskip


%gets rid of section counters 
\setcounter{secnumdepth}{0}



\section{Describing Motion:
Kinematics in One Dimension}

\subsection{Displacement}
“the change in x,” or “change in position” 

$\Delta x = x_2 - x_1$

The change in any quantity means the final value of that quantity, minus the initial value.

\subsection{Average Speed}
is defined as the total distance travelled along its path divided by the time it takes to travel this distance.

$\text{average speed} = \frac{\text{total distance travelled}}{\text{time elapsed}}$

\subsection{Average Velocity}
is defined in terms of displacement, rather than total distance travelled.

$\text{average velocity} = \frac{ \text{displacement} }{ \text{time elapsed} } = \frac{ \text{final position} - \text{initial position} }{ \text{time elapsed} }$


is defined as the displacement divided by the elapsed time

$\bar{v} = \frac{x_2 - x_1}{t_2 - t_1} = \frac{\Delta x}{\Delta t}$


\subsection{Instantaneous Velocity}
Instantaneous velocity at any moment is defined as the average velocity over an infinitesimally short time interval.

$v = \lim_{\Delta t \to 0} \frac{\Delta x}{\Delta t}$

\subsection{Average Acceleration}
is defined as the change in velocity divided by the time taken to make this change.

$\text{average acceleration} = \frac{\text{change of velocity}}{\text{time elapsed}}$

$\bar{a} = \frac{v_2 - v_1}{t_2 - t_1} = \frac{\Delta v}{\Delta t}$

\subsection{Instantaneous Acceleration}
Instantaneous acceleration, $a$, can be defined in analogy to instantaneous
velocity as the average acceleration over an infinitesimally short time interval at
a given instant.

$a = \lim_{\Delta t \to 0} \frac{\Delta v}{\Delta t}$


\begin{tcolorbox}[enhanced jigsaw,sharp corners,coltext=black,colback=BurntOrange!25!white,boxrule=0pt,breakable,size=minimal]
\subsection{Deceleration}
We have a deceleration whenever the magnitude of the velocity is decreasing;

thus the velocity and acceleration point in opposite directions when there is deceleration.







\subsection{Motion at Constant Acceleration}

We now examine motion in a straight line when the magnitude of the acceleration is constant. In this case, the instantaneous and average accelerations are equal.


We use the definitions of average velocity and acceleration to derive a set of valuable equations that relate $x$, $v$, $a$, and $t$ when $a$ is constant, allowing us to determine any one of these variables if we know the others. We can then solve many interesting Problems.

First we choose the initial time in any discussion to be zero, and we call it $t_0$.
That is,  $t_1 = t_0 = 0$. (This is effectively starting a stopwatch at $t_0$ .)
We can then let $t_2 = t$ be the elapsed time.

The initial position $x_1$ and the initial velocity $v_1$  of an object will now be represented by $x_0$ and $v_0$ , since they represent $x$ and $v$ at $t = 0$.
At time $t$ the position and velocity will be called $x$ and $v$ (rather than $x_2$ and $v_2$).
The average velocity during the time interval $t - t_0$ will be

$\bar{v} = \frac{\Delta x}{\Delta t} = \frac{x - x_0}{t - t_0} = \frac{x - x_0}{t}$

since we chose $t_0 = 0$


$a = \frac{v - v_0}{t}$


The velocity of an object after any elapsed time t

$v = v_0 + at$

Calculating the 
Position x of an object after a time t

$\bar{v} = \frac{x - x_0}{t}$

becomes

$x = x_0 + \bar{v}t$


Because the velocity increases at a uniform rate $\bar{v}$ will be midway between the initial and final velocities

$\bar{v} = \frac{v_0 + v}{2}$


Combining the last three equations these become

$x = x_0 + \bar{v}t$

$x = x_0 + (\frac{v_0 + v}{2})t$

$x = x_0 + (\frac{v_0 + (v_0 + at)}{2})t$

$x = x_0 + v_0t + \frac{1}{2}at^2$



Three of the four most useful equations for motion at constant acceleration 

The velocity of an object after any elapsed time t

$v = v_0 + at$

Average velocity

$\bar{v} = \frac{v_0 + v}{2}$

Position x of an object after a time t

$x = x_0 + v_0t + \frac{1}{2}at^2$



Situations where time $t$ is not known

$x = x_0 + (\frac{v_0 + v}{2})t$


solve for $t$
$v = v_0 + at$

$t = \frac{v - v_0}{a}$

Substituting this into the previous equation we get 

$x = x_0 + (\frac{v_0 + v}{2})(\frac{v - v_0}{a}) = x_0 + \frac{v^2 - v_0^2}{2a}$

solve for $v^2$

$v^2 = v_0^2 + 2a(x - x_0)$








Kinematic equations for constant acceleration 

$[a = \text{constant}]$

$v = v_0 + at$

$x = x_0 + v_0t + \frac{1}{2}at^2$

$v^2 = v_0^2 + 2a(x - x_0)$

$\bar{v} = \frac{v + v_0}{2}$



\subsection{Solving Problems}

1. Read and reread the whole problem carefully before trying to solve it.  

2. Decide what object (or objects) you are going to study, and for what time interval. You can often choose the initial time to be t = 0.  

3. Draw a diagram or picture of the situation, with coordinate axes wherever applicable. [You can place the origin of coordinates and the axes wherever you like to make your calculations easier. You also choose which direction is positive and which is negative. Usually we choose the x axis to the right as positive.]  

4. Write down what quantities are “known” or “given,” and then what you want to know. Consider quantities both at the beginning and at the end of the chosen time interval. You may need to “translate” language into physical terms, such as “starts from rest” means v0 = 0.  

5. Think about which principles of physics apply in this problem. Use common sense and your own experiences. Then plan an approach.  

6. Consider which equations (and/or definitions) relate the quantities involved. Before using them, be sure their range of validity includes your problem (for example, Eqs. 2–11 are valid only when the acceleration is constant). If you find an applicable  equation that involves only known quantities and one desired unknown, solve the equation algebraically for the unknown. Sometimes several sequential calculations, or a combination of equations, may be needed. It is often preferable to solve algebraically for the desired unknown before putting in numerical values.  

7. Carry out the calculation if it is a numerical problem. Keep one or two extra digits during the calculations, but round off the final answer(s) to the correct number of significant figures (Section 1–4).  

8. Think carefully about the result you obtain: Is it reasonable? Does it make sense according to your own intuition and experience? A good check is to do a rough estimate using only powers of 10, as discussed in Section 1–7. Often it is preferable to do a rough estimate at the start of a numerical problem because it can help you focus your attention on finding a path toward a solution.  

9. A very important aspect of doing problems is keeping track of units. An equals sign implies the units on each side must be the same, just as the numbers must. If the units do not balance, a mistake has been made. This can serve as a check on your solution (but it only tells you if you’re wrong, not if you’re right). Always use a consistent set of units.






\subsection{Freely Falling Objects}

with upwards as positive

solving for t

$y = y_0 + v_0t + \frac{1}{2}at^2$

We rewrite our y equation just above in standard form,

$at^2 + bt + c = 0$

$0 = (y_0 - y) + v_0t + \frac{1}{2}at^2$

$\frac{1}{2}at^2 + v_0t + (y_0 - y) = 0$

$(\frac{1}{2}a)t^2 + (v_0)t + (y_0 - y) = 0$

Using the quadratic formula, we find as solutions

$t=\frac{-b\pm\sqrt{b^2-4ac}}{2a}$

$g = 9.80 m/s^2$


\subsection{Stuff about graphing}


\subsection{Right angled triangles}

$\sin\theta=Opp/Hyp$
	
$\cos\theta=Adj/Hyp$	
	
$\tan\theta=Opp/Adj$	


$\csc\theta=Hyp/Opp=1/\sin\theta$

$\sec\theta=Hyp/Adj=1/\cos\theta$

$\cot\theta=Adj/Opp=1/\tan\theta$


\subsection{Resolving vectors}

$\mathbf{v}_x=r\cos{\theta}$

$\mathbf{v}_y=r\sin{\theta}$

$r=\left|\mathbf{v}\right|=\sqrt{x^2+y^2}$ (Magnitude)

$r=\left|\mathbf{v}\right|=\sqrt{x^2+y^2+z^2}$ (Magnitude for 3-space)

$\theta=\tan^{-1}{(\frac{\mathbf{v}_y}{\mathbf{v}_x})}$



$\theta = $ \begin{tabu}{|c|c|}
    \hline
$180 - \theta$ & $\theta$ \\ \hline
$180 + \theta$ & $360 - \theta$ \\ \hline
    \end{tabu}???

    
    
    
    
    
    
    
    
\section{Kinematics in
Two Dimensions; Vectors}
    
    
\subsection{Solving Projectile Motion Problems}

General Kinematic Equations for Constant Acceleration
in Two Dimensions


\begin{tabu}{XX}
x component (horizontal) & 
y component (vertical) \\
\hline
$v_x = v_{x0} + a_xt$ &
$v_y = v_{y0} + a_yt$ \\

$x = x_0 + v_{x0}t + \frac{1}{2}a_xt^2$ & 
$y = y_0 + v_{y0}t + \frac{1}{2}a_yt^2$ \\

$v_x^2 = v_{x0}^2 + 2a_x(x - x_0)$ & 
$v_y^2 = v_{y0}^2 + 2a_y(y - y_0)$ \\ \hline
\end{tabu}
\end{tcolorbox}


\begin{tcolorbox}[enhanced jigsaw,sharp corners,coltext=black,colback=BurntOrange!25!white,boxrule=0pt,breakable,size=minimal]
We can simplify Eqs to use for projectile motion because we can set

$a_x = 0$, $a_y = -g$


Kinematic Equations for Projectile Motion

(y positive upward; $a_x = 0, a_y = -g = 9.80 m/s^2$)




\begin{tabu}{XX}
Horizontal Motion & 
Vertical Motion \\
($a_x = 0, v_x = \text{constant}$) &
($a_y = -g = \text{constant}$) \\
\hline
$v_x = v_{x0}$ &
$v_y = v_{y0} - gt$ \\

$x = x_0 + v_{x0}t $ & 
$y = y_0 + v_{y0}t - \frac{1}{2}gt^2$ \\

 & 
$v_y^2 = v_{y0}^2 - 2g(y - y_0)$ \\ \hline
\end{tabu}







\end{tcolorbox}


\begin{tcolorbox}[enhanced jigsaw,sharp corners,coltext=black,colback=BurntOrange!25!white,boxrule=0pt,breakable,size=minimal]
\subsection{Projectile Motion Is Parabolic}

 We now show that the path followed by any projectile is a parabola, if we can ignore air resistance and can assume that g is constant. 
 
 for simplicity we set x0 = y0 = 0 
 
 
$x = v_{x0}t $

$y = v_{y0}t - \frac{1}{2}gt^2$ 
 
From the first equation, we have $t = x/v_{x0}$ , and we substitute this into the second one to obtain
 
$y = v_{y0}(\frac{x}{v_{x0}}) - \frac{1}{2}g(\frac{x}{v_{x0}})^2$ 

$y = (\frac{v_{y0}}{v_{x0}})x - \frac{1}{2}g(\frac{x^2}{v_{x0}^2})$ 

$y = (\frac{v_{y0}}{v_{x0}})x - (\frac{g}{2v_{x0}^2})x^2$ 

We see that $y$ as a function of $x$ has the form $y = Ax - Bx^2$, where $A$ and $B$ are constants for any specific projectile motion. This is the standard equation for a parabola.
\end{tcolorbox}


\section{Latex tools}

https://www.tablesgenerator.com/latex\_tables

http://w2.syronex.com/jmr/latex-symbols-converter




\iftoggle{editing}{%
  % editing
}{%
  % finaloutput
  \end{multicols}
}







\end{document}
